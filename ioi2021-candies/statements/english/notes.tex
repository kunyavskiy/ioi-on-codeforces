\textbf{Example 1}

Consider the following call:

\texttt{distribute\_candies([10, 15, 13], [0, 0], [2, 1], [20, -11])}

This means that box $0$ has a capacity of $10$ candies, box $1$ has a capacity of $15$ candies, and box $2$ has a capacity of $13$ candies.

At the end of day $0$, box $0$ has $\min(c[0], 0+v[0]) = 10$ candies, box $1$ has
$\min(c[1], 0+v[0])=15$ candies and box $2$ has $\min(c[2], 0+v[0])=13$ candies.

At the end of day $1$, box $0$ has $\max(0, 10+v[1]) = 0$ candies, box $1$ has
$\max(0, 15+v[1]) = 4$ candies. Since $2 > r[1]$, there is no change in the number of candies in box $2$. The number of candies at the end of each day are summarized below:

\begin{center}
\renewcommand{\arraystretch}{1.5}
\begin{tabular}{|c|c|c|c|}
\hline
 Day & Box $0$ & Box $1$ & Box $2$ \\
\hline
$0$ & $10$ & $15$ & $13$ \\
\hline
$1$ & $0$ & $4$ & $13$ \\
\hline

\end{tabular}
\end{center}

As such, the procedure should return $[0, 4, 13]$