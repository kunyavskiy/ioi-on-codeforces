Тетя Хонг готовит $n$ коробок конфет для учеников ближайшей школы.
Коробки пронумерованы от $0$ до $n-1$ и изначально пустые.
Коробка номер $i$ ($0 \leq i \leq n-1$) может вместить в себя $c[i]$ конфет.

Тетя Хонг потратит $q$ дней на приготовление коробок.
На $j$-й день ($0 \leq j \leq q-1$), ее действия определяются тремя числами $l[j]$, $r[j]$ и $v[j]$, где $0 \leq l[j] \leq r[j] \leq n-1$ и $v[j] \neq 0$. А именно, для каждой коробки с номером $k$, где $l[j] \leq k \leq r[j]$, происходит следующее:
\begin{itemize}
\item Если $v[j] > 0$, Тетя Хонг добавляет в коробку номер $k$ конфеты по одной до тех пор, пока она не добавит ровно $v[j]$ конфет или пока коробка не станет полностью заполненной. Другими словами, если коробка содержала $p$ конфет, то после этой операции она будет содержать $\min(c[k],p+v[j])$ конфет.

\item Если $v[j] < 0$, Тетя Хонг убирает конфеты из коробки номер $k$ по одной до тех пор, пока она не уберет ровно $-v[j]$ конфет или пока коробка не станет пустой. Другими словами, если коробка содержала $p$ конфет, то после этой операции она будет содержать $\max(0,p+v[j])$ конфет.
\end{itemize}

Необходимо определить количество конфет в каждой из коробок после $q$ дней.

\textbf{Детали реализации}

Вам необходимо реализовать следующие функции:
\begin{itemize}
\item  \texttt{int[] distribute\_candies(int[] c, int[] l, int[] r, int[] v)}
\begin{itemize}
\item $c$: массив длины $n$. Для всех $0 \leq i \leq n-1$ число $c[i]$ означает вместимость коробки с номером $i$.
\item $l$, $r$ и $v$: три массива длины $q$. На день $j$, для всех $0 \leq j \leq q-1$, действия Тети Хонг определяются числами $l[j]$, $r[j]$ и $v[j]$, согласно описанию выше.
\item Функция должна вернуть массив длины $n$. Обозначим его за $s$. Для каждого $0 \leq i \leq n-1$ число $s[i]$ должно быть равно количеству конфет в коробке номер $i$ после $q$ дней.
\end{itemize}
\end{itemize}
