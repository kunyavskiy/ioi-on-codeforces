\textbf{Пример 1}

Рассмотрим следующий вызов функции:

\texttt{distribute\_candies([10, 15, 13], [0, 0], [2, 1], [20, -11])}
	
В этом примере коробка $0$ имеет вместимость $10$ конфет, коробка $1$ имеет вместимость $15$ конфет, а коробка $2$ имеет вместимость $13$ конфет.

В конце дня $0$ в коробке $0$ находится $\min(c[0], 0+v[0]) = 10$ конфет, в коробке $1$~--- $\min(c[1], 0+v[0])=15$ конфет, а в коробке $2$~--- $\min(c[2], 0+v[0])=13$ конфет.

В конце дня $1$ в коробке $0$ находится $\max(0, 10+v[1]) = 0$ конфет, в коробке $1$~--- $\max(0, 15+v[1]) = 4$ конфеты. Так как $2 > r[1]$, количество конфет в коробке номер $2$ не изменится. Количество конфет в каждой коробке в конце каждого дня представлено в таблице ниже:
\begin{center}
\renewcommand{\arraystretch}{1.5}
\begin{tabular}{|c|c|c|c|}
\hline
 День & Коробка $0$ & Коробка $1$ & Коробка $2$ \\
\hline
$0$ & $10$ & $15$ & $13$ \\
\hline
$1$ & $0$ & $4$ & $13$ \\
\hline

\end{tabular}
\end{center}

Таким образом, функция должна вернуть $[0, 4, 13]$.

