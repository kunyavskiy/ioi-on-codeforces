В первой строке записано единственное целое число $F$ "--- начальное
количество рыбок в озере ($1 \le F \le 500\,000$). Во второй
строке записано единственное целое число $K$ ($1 \le K \le F$)
"--- количество видов драгоценных камней (пронумерованных целыми числами от
$1$ до $K$). В третьей строке записано единственное целое число $M$ "---
модуль ($2 \le M \le 30\,000$). Каждая из следующих $F$ строк
содержит по два целых числа "--- длину соответствующей рыбки $L_i$ и номер
вида драгоценного камня $G_x$, проглоченного ей
($1 \le L_x \le 1\,000\,000\,000$, $1 \le G_x \le K$).

Гарантируется, что хотя бы один камень каждого вида от $1$ до $K$ проглочен
хотя бы одной рыбкой.
