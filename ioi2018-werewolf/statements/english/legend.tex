There are cities $N$ and $M$ roads in Ibaraki Prefecture, Japan. Cities are numbered from $0$ through $N-1$ in the increasing order of their population. Each road connects a pair of distinct cities, and can be traveled in both directions. You can travel from any city to any other city by using one or more of these roads.

You planned $Q$ trips, numbered from $0$ through $Q-1$. The trip $i$ $(0 \le i \le Q-1)$ is to travel from $S_i$ the city to the city $E_i$.

You are a werewolf. You have two forms: \bf{human form} and \bf{wolf form}. At the beginning of each trip you are in human form. At the end of each trip, you must be in wolf form. During the trip you have to \bf{transform} (change from human form to wolf form) exactly once. You can transform only when you are in some city (possibly $S_i$ or $E_i$).

Living as a werewolf is not easy. You must avoid low-populated cities when you are in human form, and avoid highly-populated cities when you are in wolf form. For each trip $i$ $(0 \le i \le Q-1)$ there are two thresholds $L_i$ and $R_i$ $(0 \le L_i \le R_i \le N-1)$ that indicate which cities must be avoided. More specifically, you must avoid the cities $0, 1,  \dots , L_i-1$ when you are in human form, and must avoid the cities $R_i+1, R_i+2, \dots ,N-1$ when you are in wolf form. This means in the trip $i$, you can only transform in one of the cities $L_i, L_i+1, \dots , R_i.$

Your task is to determine, for each trip, whether it is possible to travel from the city $S_i$ to the city $E_i$ in a way that satisfies the aforementioned constraints. The route you take can have an arbitrary length.

\bf{Implementation details}

You should implement the following function:

\t{int[] check_validity(int N, int[] X, int[] Y, int[] S, int[] E, int[] L, int[] R)}

\begin{itemize}
\item $N$: the number of cities. 
\item $X$ and $Y$: arrays of length $M$. For each $j$ $(0 \le j \le M-1)$, the city $X[j]$ is directly connected to the city $Y[j]$ by a road. 
\item $S, E, L,$ and $R$: arrays of length $Q$, representing the trips.
\end{itemize}

Note that the values of $M$ and $Q$ are the lengths of the arrays, and can be obtained as indicated in the implementation notice.

The function \t{check_validity} is called exactly once for each test case. This function should return an array $A$ of integers of length $Q$. The value of $A_i$ $(0 \le i \le Q-1)$ must be $1$ if the trip $i$ is possible while satisfying the aforementioned conditions, or $0$ otherwise.


\bf{Constraints}

\begin{itemize}
\item $2 \le N \le 200\,000$
\item $N-1 \le M \le 400\,000$
\item $1 \le Q \le 200\,000$
\item For each $j : 0 \le j \le M-1$
\begin{itemize}
\item $0 \le X_j \le N-1$
\item $0 \le Y_j \le N-1$
\item $X_j \neq Y_j$
\end{itemize}
\item You can travel from any city to any other city by using roads. 
\item Each pair of cities are directly connected by at most one road. In other words, $(X_j, Y_j) \neq (X_k, Y_k)$ and $(Y_j, X_j) \neq (X_k, Y_k)$ for all $j, k: 0 \le j < k \le M-1$
\item For each $i: 0 \le i \le Q-1$
\begin{itemize}
\item $0 \le L_i \le S_i \le N-1$
\item $0 \le E_i \le R_i \le N-1$
\item $S_i \neq E_i$
\item $L_i \le R_i$
\end{itemize}
\end{itemize}

\bf{Sample grader}

The sample grader reads the input in the following format:

\begin{tabular}{clclcr}
line&$1$&:&$N$ $M$ $Q$&&\\
line&$2+j$&:&$X_j$ $Y_j$&&$0 \le j \le M-1$\\
line&$2+M+i$&:&$S_i$ $E_i$ $L_i$ $R_i$ &&$0 \le i \le Q-1$\\
\end{tabular}

The sample grader prints the return value of \t{check_validity} in the following format: 

\begin{tabular}{clclcr}
line&$1+i$&:&$A_i$&&$0 \le i \le Q-1$
\end{tabular}
