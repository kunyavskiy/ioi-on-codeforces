В префектуре Ибараки в Японии $N$ городов и $M$ дорог. Города пронумерованы от $0$ до $N-1$ по возрастанию их численности. Каждая дорога соединяет пару различных городов, и по ней можно перемещаться в обоих направлениях. Из любого города в любой другой можно добраться, переместившись по одной или нескольким дорогам.

Вы запланировали $Q$ путешествий, пронумерованных от $0$ до $Q-1$. Путешествие $i$ $(0 \le i \le Q-1)$ должно начаться в городе $S_i$ и закончиться в городе $E_i$.

Вы --- оборотень. Вы можете принимать две формы: \bf{форму человека} и \bf{форму волка}. В начале каждого путешествия вы находитесь в форме человека, а в конце каждого путешествия вы должны быть в форме волка. Во время путешествия вам необходимо \bf{превратиться} (сменить форму человека на форму волка) ровно один раз. Вы можете превращаться, только находясь в каком-либо городе (возможно, в городе $S_i$ или $E_i$).

Жизнь оборотня непроста. В форме человека вам необходимо избегать малонаселённых городов, а в форме волка --- густонаселённых городов. Для каждого путешествия $i$ $(0 \le i \le Q-1)$ заданы два числа $L_i$ и $R_i$ $(0 \le L_i \le R_i \le N-1)$, которые описывают, какие города вам необходимо избегать. Более конкретно, необходимо избегать города $0, 1,  \dots , L_i-1,$ находясь в форме человека, а также города $R_i+1, R_i+2, \dots ,N-1,$ находясь в форме волка. В частности, это означает, что вам необходимо превратиться в одном из городов $L_i, L_i+1, \dots , R_i.$

Ваша задача --- для каждого путешествия определить, возможно ли добраться из города $S_i$ в город $E_i$, чтобы описанные выше ограничения выполнялись. Ваш путь может иметь произвольную длину.

\bf{Детали реализации}

Вам требуется реализовать следующую функцию:

\t{int[] check_validity(int N, int[] X, int[] Y, int[] S, int[] E, int[] L, int[] R)}

\begin{itemize}
\item $N$: количество городов. 
\item $X$ и $Y$: массивы длины $M$. Для каждого $j: (0 \le j \le M-1),$ города $X[j]$ и $Y[j]$ напрямую связаны дорогой. 
\item $S, E,$ $L$ и $R$: массивы длины $Q$, описывающие путешествия.
\end{itemize}

Обратите внимание, что значения $M$ и $Q$ --- это длины массивов, способ получения которых описан в памятке о деталях реализации.

Функция \t{check_validity} вызывается ровно один раз для каждого теста. Функция должна вернуть массив A из $Q$ целых чисел. Значение $A_i$ $(0 \le i \le Q-1)$ должно быть равно $1$, если путешествие можно совершить согласно описанным ограничениям, и $0$ в противном случае.


\bf{Ограничения}

\begin{itemize}
\item $2 \le N \le 200\,000$
\item $N-1 \le M \le 400\,000$
\item $1 \le Q \le 200\,000$
\item Для каждого $j : 0 \le j \le M-1$
\begin{itemize}
\item $0 \le X_j \le N-1$
\item $0 \le Y_j \le N-1$
\item $X_j \neq Y_j$
\end{itemize}
\item От любого города до любого другого можно добраться по дорогам.
\item Каждая пара городов напрямую связана не более, чем одной дорогой. Другими словами, для всех пар индексов $j, k: 0 \le j < k \le M-1$ верно $(X_j, Y_j) \neq (X_k, Y_k)$ и $(Y_j, X_j) \neq (X_k, Y_k)$
\item Для каждого $i: 0 \le I \le Q-1$
\begin{itemize}
\item $0 \le L_i \le S_i \le N-1$
\item $0 \le E_i \le R_i \le N-1$
\item $S_i \neq E_i$
\item $L_i \le R_i$
\end{itemize}
\end{itemize}


\bf{Sample grader}

Пример проверяющего модуля считывает ввод в следующем формате:

\begin{tabular}{clclcr}
строка&$1$&:&$N$ $M$ $Q$&&\\
строка&$2+j$&:&$X_j$ $Y_j$&&$0 \le j \le M-1$\\
строка&$2+M+i$&:&$S_i$ $E_i$ $L_i$ $R_i$ &&$0 \le i \le Q-1$\\
\end{tabular}

Пример проверяющего модуля выводит значение, возвращаемое функцией \t{check_validity}, в следующем формате:

\begin{tabular}{clclcr}
строка&$1+i$&:&$A_i$&&$0 \le i \le Q-1$
\end{tabular}
