Джан-Джи планирует провести свой следующий отпуск в Тайване. Во время отпуска он
собирается переезжать из города в город и посещать достопримечательности в этих городах.

В Тайване $n$ городов, которые расположены вдоль единственной магистрали. Города
пронумерованы последовательно целыми числами от $0$ до $n - 1$. Для $i$-го города, $0 < i < n - 1$, соседними являются города с номерами $i - 1$ и $i + 1$. Город с номером $0$
соседствует только с городом с номером $1$, а город с номером $n - 1$ соседствует только с
городом с номером $n - 2$.

В каждом городе содержится некоторое количество достопримечательностей. Джан-Джи
планирует посетить как можно больше достопримечательностей во время своего отпуска
продолжительностью $d$ дней. Он уже выбрал город, с которого начнет отпуск. Каждый день
Джан-Джи может либо переехать из текущего города в один из соседних, либо посетить все
достопримечательности в городе, в котором он находится. Он не может сделать оба действия
в один день. Джан-Джи \texttt{никогда не посещает достопримечательность дважды в одном
городе}, даже если приезжает в этот город несколько раз. Пожалуйста, помогите Джан-Джи
спланировать отпуск так, чтобы он посетил как можно больше достопримечательностей.

\textbf{Постановка задачи}

Вам требуется реализовать функцию \texttt{findMaxAttraction}, которая вычисляет максимальное количество достопримечательностей, которые Джан-Джи может посетить.

\begin{itemize}
\item \texttt{long long int findMaxAttraction(int n, int start, int d,
int attraction[])}
\begin{itemize}
\item $n$: количество городов.
\item $start$: номер начального города.
\item $d$: количество дней.
\item $attraction$: массив длины $n$; $attraction[i]$ задает количество
достопримечательностей в $i$-м городе, для $0 \le i \le n - 1$.
\item функция должна возвращать максимальное количество достопримечательностей,
которые Джан-Джи может посетить.
\end{itemize}
\end{itemize}
