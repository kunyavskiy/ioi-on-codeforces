You have to submit exactly one file, called \texttt{holiday.cpp}. This file should implement the subprogram described above using the following signatures. You also need to
include a header file \texttt{holiday.h} for C/C++ implementation.

Suppose Jian-Jia has 7 days of holiday, there are 5 cities (listed in the table below), and he starts from city 2. On the first day Jian-Jia visits the 20 attractions in city 2. On the second day Jian-Jia moves from city 2 to city 3, and on the third day visits the 30 attractions in city 3. Jian-Jia then spends the next three days moving from city 3 to city 0, and visits the 10 attractions in city 0 on the seventh day. The total number of attractions Jian-Jia visits is $20 + 30 + 10 = 60$, which is the maximum number of
attractions Jian-Jia can visit in 7 days when he starts from city 2.

\begin{center}
\renewcommand{\arraystretch}{1.5}
\begin{tabular}{|c|c|}
\hline
City & Number of attractions \\
\hline
0 &  10 \\
\hline
1 &  2 \\
\hline
2 & 20 \\
\hline
3 & 30 \\
\hline
4 & 1 \\
\hline
\end{tabular}
\end{center}

\begin{center}
\renewcommand{\arraystretch}{1.5}
\begin{tabular}{|c|c|}
\hline
Day & Action \\
\hline
1 & visit the attractions in city 2 \\
\hline
2 & move from city 2 to city 3 \\
\hline
3 & visit the attractions in city 3 \\
\hline
4 & move from city 3 to city 2 \\
\hline
5 & move from city 2 to city 1 \\
\hline
6 & move from city 1 to city 0 \\
\hline
7 & visit the attractions in city 0 \\
\hline
\end{tabular}
\end{center}