% Origin: IOI 2004
% Translation: Ekaterina Mukoseeva

Зевс подарил Артемиде, богине охоты, прямоугольный участок земли,
чтобы вырастить лес.
Левая сторона этого участка "--- это отрезок неотрицательного
луча оси ординат, нижняя сторона "--- отрезок неотрицательного луча
оси абсцисс, а $(0, 0)$ "--- левый нижний угол участка.
Зевс сказал Артемиде выращивать деревья только в точках с целыми координатами.

Артемида любит, когда лес выглядит естественно,
и поэтому в её лесу любая прямая, проходящая через два дерева,
не параллельна ни оси абсцисс, ни оси ординат.

Иногда Зевс хочет, чтобы Артемида вырубила для него деревья.
Деревья должны быть вырублены так, чтобы выполнялись следующие условия:

\begin{itemize}
\item
Зевс хочет, чтобы для него было вырублено хотя бы $T$ деревьев.
\item
Чтобы потом построить прямоугольное футбольное поле для будущих
футбольных успехов, Артемида должна вырубить все деревья внутри
и на границе некоторой прямоугольной площадки, и ни одно дерево вне
этой площадки.
\item
Стороны этой площадки должны быть параллельны осям координат.
\item
В двух противоположных углах площадки должны быть расположены деревья
(которые будут вырублены).
\end{itemize}

Так как Артемида любит деревья, она хочет выполнить все условия таким образом,
чтобы пришлось вырубить как можно меньше деревьев.
Напишите программу, которая по информации о лесе и минимальному количеству
деревьев $T$, которые должны быть вырублены, выбирает площадку,
на которой Артемида будет вырубать деревья.

