\textbf{Пример 1}

Рассмотрим сценарий, в котором собрано $3$ гриба типов $[A, B, B]$, в этом порядке.
Функция \t{count\_mushrooms} будет вызвана следующим образом:

\t{count\_mushrooms(3)}

Эта функция может вызвать, например, \t{use\_machine([0, 1, 2])}, этот вызов вернет $1$.

После этого она может, например, вызвать \t{use\_machine([2, 1])}, этот вызов вернет $0$.

Теперь достаточно информации, чтобы сделать вывод, что имеется только $1$ гриб типа A.
Таким образом, функция \t{count\_mushrooms} должна вернуть $1$.

\textbf{Пример 2}

Рассмотрим пример, в котором собрано $4$ гриба типов $[A, B, A, A]$, в этом порядке.
Функция \t{count\_mushrooms} будет вызвана следующим образом:

\t{count\_mushrooms(4)}

Эта функция может вызвать, например, \t{use\_machine([0, 2, 1, 3])},
 этот вызов вернет $2$.
После этого она может, например, вызвать \t{use\_machine([1, 2])},
 этот вызов вернет $1$.

Теперь достаточно информации, чтобы сделать вывод, что имеется $3$ гриба типа A.
Таким образом, функция \t{count\_mushrooms} должна вернуть $3$.