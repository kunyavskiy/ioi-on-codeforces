В игре под названием <<Память>> используется $50$ карт. Каждая из букв от <<А>> до <<Y>> (\t{ASCII}-коды которых соответственно равны от $65$ до $89$) напечатана на лицевой стороне ровно двух карт. Перед игрой все карты перемешиваются случайным образом и выкладываются на стол лицевой стороной вниз.

Во время игры Джек каждый раз переворачивает две карты лицевой стороной вверх и видит две буквы. Джек получает от мамы конфету за каждую из $25$ букв тогда, когда он в первый раз увидит одновременно эту букву на обеих перевернутых картах. Например, когда Джек перевернул две карты и в первый раз увидел на обеих картах букву <<M>>, он получает конфету. После этого, независимо от того, равны буквы на перевернутых картах или нет, карты снова переворачиваются лицевой стороной вниз. Процесс игры повторяется до тех пор, пока Джек не получит все $25$ конфет --- по одной за каждую букву.

Вам необходимо написать процедуру \t{play}, которая играет в данную игру. Ваша процедура должна вызывать процедуру \t{faceup(C)}, которая реализована в системе оценивания, где $C$ --- целое число от $1$ до $50$, обозначающее номер конкретной карты, которую необходимо перевернуть лицевой стороной вверх. На момент вызова процедуры \t{faceup(C)} карта с номером $C$ не должна быть перевернута лицевой стороной вверх. Процедура \t{faceup(C)} возвращает символ, который обозначает букву, изображённую на карте с номером $C$.

После каждого второго вызова процедуры \t{faceup}, система оценивания автоматически переворачивает карты снова лицевой стороной вниз.

Ваша процедура \t{play} может завершить свою работу только тогда, когда Джек получил все $25$ конфет. Разрешено делать вызовы процедуры \t{faceup(C)} даже после того, как Джек получит последнюю конфету

Ниже приведен пример (вместе с комментариями) одной из возможных последовательностей вызовов, сделанных вашей процедурой \t{play}.

\begin{tabular}{|c|c|c|}
\hline
Вызов & Результат вызова & Пояснение\\ \hline
\t{faceup(1) }& `B' &  На карте $1$ написана буква `B'. \\ \hline
\t{faceup(7) }& `X' &  \parbox{10cm}{\centering \vspace{2mm}На карте $7$ написана буква `X'. Буквы не совпадают. Система оценивания автоматически переворачивает карты $1$ и $7$ лицевой стороной вниз.\\\vspace{2mm}}
 \\ \hline
\t{faceup(7) }& `X' &  На карте $7$ написана буква `X'. \\ \hline
\t{faceup(15)} & `O' & \parbox{10cm}{\centering \vspace{2mm}На карте $15$ написана буква `O'. Буквы не совпадают. Система оценивания автоматически переворачивает карты $7$ и $15$ лицевой стороной вниз.\\\vspace{2mm}}
 \\ \hline
\t{faceup(50)} & `X' & На карте $50$ написана буква `X'. \\ \hline
\t{faceup(7) }& `X' &  \parbox{10cm}{\centering \vspace{2mm}На карте $7$ написана буква `X'. Джек получает первую конфету Система оценивания автоматически переворачивает карты $50$ и $7$ лицевой стороной вниз\\\vspace{2mm}}
\\ \hline
\t{faceup(7) }& `X' &  На карте $7$ написана буква `X'. \\ \hline
\t{faceup(50)} & `X' & \parbox{10cm}{\centering \vspace{2mm}На карте $50$ написана буква `X'. Буквы совпадают,
но Джек не получает конфету. Система оценивания автоматически переворачивает карты  $7$ и $50$ лицевой стороной вниз. \\\vspace{2mm}}\\ \hline
\t{faceup(2) }& `B' &  На карте $2$ написана буква `B'. \\ \hline
... & ... & (Некоторые вызовы процедуры были пропущены) \\ \hline
\t{faceup(1) }& `B' &  На карте $1$ написана буква `B'. \\ \hline
\t{faceup(2) }& `B' &  На карте $2$ написана буква `B'. Джек получает 25-ю конфету. \\ \hline
\end{tabular}