Джек и Джилл играют в игру <<Горячо-холодно>>. Джилл загадывает целое число от $1$ до $N$, а Джек делает последовательные попытки угадать число.

Каждая из догадок Джека является числом от $1$ до $N$. На каждую догадку Джилл отвечает словами <<теплее>>, <<холоднее>> или <<одинаково>>. На первую догадку Джека Джилл отвечает словом <<одинаково>>. На остальные догадки Джилл отвечает словами: 
\begin{itemize}
\item <<теплее>>, если эта догадка ближе к задуманному числу, чем предыдущая; 
\item <<холоднее>>, если эта догадка дальше от задуманного числа, чем предыдущая; 
\item <<одинаково>>, если эта догадка не ближе и не дальше от задуманного числа, чем предыдущая;
\end{itemize}

Ваша задача --- написать процедуру \t{HC(N)}, которая играет за Джека. Эта процедура может периодически вызывать процедуру \t{Guess(G)}, где $G$ --- число от $1$ до $N$. Процедура \t{Guess(G)} будет возвращать значение $1$ при ответе <<теплее>>, значение $-1$ при ответе <<холоднее>> или значение $0$ при ответе <<одинаково>>. Процедура \t{HC(N)} должна вернуть задуманное число

Для примера предположим, что $N=5$, и Джилл задумала число $2$. Если процедура \t{HC} сделает следующую последовательность вызовов , то будут возвращены значения, показанные во втором столбце таблицы ниже

\begin{tabular}{|l|l|l|} \hline
Вызов & Возвращаемое значение & Пояснение\\ \hline
\t{Guess(5)} & $0$ & Одинаково (первый вызов) \\ \hline
\t{Guess(3)} & $1$ & Теплее \\ \hline
\t{Guess(4)} & $-1$ & Холоднее \\ \hline
\t{Guess(1)} & $1$ & Теплее \\ \hline
\t{Guess(3)} & $0$ & Одинаково \\ \hline
\end{tabular}

В этот момент Джек знает задуманное число, и процедура \t{HC} должна вернуть $2$. Джек затратил $5$ догадок, чтобы определить задуманное число. Вы можете сделать лучше.
