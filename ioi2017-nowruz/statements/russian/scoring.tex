Выходной файл считается \texttt{корректным}, если выполнены следующие условия:
\begin{itemize}
\item Карта лабиринта на выходе должна совпадать с картой сада на входе за единственным
исключением: произвольное количество символов `.' (свободная клетка) может быть
изменено на символы `X' (клетка, занятая кустом).
\item Карта на выходе должна описывать лабиринт согласно определению, данному в условии
задачи.
\end{itemize}

Если выходной файл для теста не является корректным, вы получаете за этот тест $0$ баллов.
В противном случае, количество баллов равно $\min(10, \frac{10 \cdot l}{k})$ с округлением вниз до второго знака после запятой. В формуле выше $l$ означает количество детей, которые могут спрятаться в выходном лабиринте, а $k$ означает число, данное во входном файле. $10$ баллов за тест начисляется лишь в том случае, когда в лабиринте могут спрятаться $k$ или более детей. Для каждого теста существует решение, которое набирает $10$ баллов.

Обратите внимание, что если решение корректно, но получает $0$ баллов согласно формуле
выше, в системе CMS будет отображаться вердикт <<Неправильный ответ>>.