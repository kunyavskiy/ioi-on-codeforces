Рассмотрим следующий тест:

\begin{verbatim}
4 5 5
....#
.#..#
...#.
....#
\end{verbatim}


Ниже приведён один из возможных корректных ответов:

\begin{verbatim}
.X.X#
.#..#
...#X
XX..#
\end{verbatim}

В этом лабиринте могут спрятаться $l = 4$ детей, поэтому такое решение получит баллов $10 \cdot 4 / 5 = 8$. Клетки, в которых могут спрятаться дети, ниже отмечены как `O':

\begin{verbatim}
OXOX#
.#.O#
...#X
XX.O#
\end{verbatim}

Следующие три решения не являются корректными:

\begin{verbatim}
.XXX#        ...X#        XXXX#
.#XX#        .#.X#        X#XX#
...#.        ...#X        ..X#X
XX..#        XXXX#        ..XX#
\end{verbatim}

В решении слева не существует простого пути между свободной клеткой в левом верхнем углу
и свободной клеткой в правом столбце. В двух других решениях для каждой пары различных
свободных клеток существует ровно два простых пути между этими клетками.