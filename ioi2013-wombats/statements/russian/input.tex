Грейдер читает входные данные в следующем формате:
\begin{itemize}
\item строка $1$: $R\ C$
\item строка $2$: $H[0][0]\dots H[0][C­-2]$
\item \dots
\item строка $(R + 1)$: $H[R­-1][0] \dots H[R­-1][C­-2]$
\item строка $(R + 2)$: $V[0][0] \dots V[0][C­-1]$
\item \dots
\item строка $2R$: $V[R­-2][0] \dots V[R­-2][C­-1]$
\item следующая строка: $E$
\item следующие $E$ строк: каждая строка описывает одно событие. События перечислены в том порядке, в котором они происходят.
\end{itemize}
Если $C = 1$, то пустые строки, описывающие число вомбатов на горизонтальных дорогах (строки с номерами от $2$ до $R + 1$), могут быть пропущены.

Для каждого события должен быть использован один из следующих форматов:
\begin{itemize}
\item для описания вызова функции \t{changeH(P, Q, W)}: $1\ P\ Q\ W$
\item для описания вызова функции \t{changeV(P, Q, W)}: $2\ P\ Q\ W$
\item для описания вызова функции \t{escape(V1, V2)}: $3\ V1\ V2$
\end{itemize}
