Ботаник Сомхед часто организует экскурсии групп студентов в один из самых больших тропических садов Таиланда. Этот сад состоит из $N$ фонтанов, пронумерованных $0, 1, \ldots, N-1$, и $M$ тропинок. Каждая тропинка соединяет пару различных фонтанов, и по ней можно ходить в обоих направлениях. При этом различные тропинки соединяют различные пары. От каждого фонтана ведет хотя бы одна тропинка. Вдоль каждой из тропинок находятся красивые коллекции растений, которые хотел бы увидеть Сомхед. Каждая группа может начать свою экскурсию от любого из фонтанов.

Сомхед любит тропические растения. Поэтому от каждого из фонтанов он и его группа студентов пойдут по наиболее красивой тропинке, которая ведет от этого фонтана, за одним исключением: если самой красивой является та тропинка, по которой они только что прошли. В этом случае, если есть другие тропинки, идущие от этого фонтана, то Сомхед и его группа пойдут по второй по красоте тропинке. Конечно, если другой тропинки нет, они вернутся по той же тропинке, по которой и пришли, используя эту тропинку второй раз подряд. Так как Сомхед~--- профессиональный ботаник, то среди тропинок нет двух одинаково красивых для него.

Студенты не очень интересуются растениями. Однако им бы хотелось пообедать в шикарном ресторане, который находится у фонтана с номером $P$. Сомхед знает, что его студенты проголодаются после того, как пройдут ровно $K$ тропинок, где $K$ может быть различным для разных групп студентов. Сомхеда интересует, сколько различных маршрутов он может выбрать для каждой из групп студентов, если:

\begin{itemize}
\item каждая группа может начинать прогулку от любого фонтана; 
\item последующая тропинка всегда выбирается описанным выше способом; 
\item каждая группа должна прийти к фонтану с номером $P$, пройдя ровно по $K$ тропинкам.
\end{itemize}

Необходимо заметить, что по пути группа студентов может проходить мимо фонтана с номером $P$, следуя по своему маршруту, тем не менее, маршрут обязательно должен заканчиваться у фонтана с номером $P$.

По данной информации о фонтанах и тропинках, необходимо найти количество различных маршрутов для каждой из $Q$ групп студентов, то есть, $Q$ значений $K$.

Вам необходимо написать процедуру \t{count\_routes(N,M,P,R,Q,G)} которая принимает следующие параметры: 
\begin{itemize}
\item $N$~--- количество фонтанов. Фонтаны нумеруются от $0$ до $N-1$. 
\item $M$~--- количество тропинок. Тропинки нумеруются от $0$ до $M-1$. Тропинки заданы в порядке убывания красоты: для $0 \le i < M-1$, тропинка с номером $i$ более красива, чем тропинка с номером $i+1$. 
\item $P$~--- номер фонтана, возле которого находится шикарный ресторан. 
\item $R$~--- двумерный массив, в котором описываются тропинки. Для $0 \le i < M$, тропинка с номером $i$ соединяет фонтаны с номерами $R[i][0]$ и $R[i][1]$. Следует учесть, что каждая тропинка соединяет различные фонтаны, и никакие две тропинки не соединяют одну и ту же пару фонтанов. 
\item $Q$~--- количество групп студентов. 
\item $G$~--- одномерный массив целых чисел, содержащий значения $K$. Для $0 \le i < Q$ в элементе массива $G[i]$ задано количество тропинок $K$, по которым должна пройти $i$-я группа.
\end{itemize}

Для $0 \le i < Q$ процедура должна найти количество возможных маршрутов с количеством тропинок, равным $G[i]$, по которым может пройти $i$-я группа, чтобы попасть к фонтану с номером $P$. Для каждой группы с номером $i$ процедура должна вызывать процедуру \t{answer(X)}, чтобы сообщить, что число маршрутов равно $X$. Ответы необходимо выдавать в том же порядке, в котором заданы группы. Если допустимых маршрутов не существует, процедура должна вызывать процедуру \t{answer(0)}.
