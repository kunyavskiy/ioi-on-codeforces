\texttt{who\_wins([0, 1], [1, 0], [0, 0, 1, 1], [0, 1, 0, 1])}

\includegraphics{train.png}

\begin{itemize}
\item В железнодорожной сети есть две станции. Борзу принадлежит станция $0$, которая
является зарядной. Арезу принадлежит станция $1$, которая не является зарядной.
\item В сети есть $4$ пути $(0, 0), (0, 1), (1, 0)$, и $(1, 1)$, где $(i, j)$ обозначает односторонний путь, идущий от станции $i$ к станции $j$.
\item Рассмотрим игру, в которой поезд исходно расположен на станции $0$.
Если Борзу установит переключатель станции $0$ на путь $(0, 0)$, поезд будет неограниченно
перемещаться по циклу, состоящему из этого пути (обратите внимание, что станция $0$ является зарядной). В этом случае выиграет Арезу. В противном случае, если Борзу
установит переключатель станции $0$ на путь $(0, 1)$, Арезу может установить
переключатель станции $1$ на путь $(1, 0)$. Поезд будет неограниченно перемещаться
между двумя станциями. Арезу выиграет, так как станция $0$ является зарядной, и поезд никогда не остановится. Таким образом, Арезу может выиграть игру вне зависимости от
ходов Борзу.
\item Аналогично можно убедиться, что в игре, начинающейся на станции $1$, Арезу также
выигрывает, вне зависимости от ходов Борзу. Таким образом, функция должна вернуть $[1, 1]$.
\end{itemize}
