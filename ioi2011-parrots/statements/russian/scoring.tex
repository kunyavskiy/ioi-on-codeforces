\begin{center}
\renewcommand{\arraystretch}{1.5}
\begin{tabular}{|c|c|c|c|l|}
\hline
Подзадача & Баллы & $N$ & $R$ & \parbox{7cm}{\centering \vspace{2mm}Дополнительные ограничения на входные данные\\\vspace{2mm}}\\
\hline
1 & 17 & $N = 8$ & $0 \le R \le 65535$ & \parbox{7cm}{\centering \vspace{2mm}Каждое целое число в массиве $M$ равно $0$ или $1$. Количество вызовов процедуры \t{send} должно быть не более $K=10 \cdot N$.\\\vspace{2mm}}\\
\hline
2 & 17 & $1 \leq N \leq 16$ & $0 \le R \le 65535$ &\parbox{7cm}{\centering \vspace{2mm} Количество вызовов процедуры \t{send} должно быть не более $K=10 \cdot N$.\\\vspace{2mm}} \\
\hline
3 & 18 & $1 \leq N \leq 16$ & $0 \le R \le 255$ & \parbox{7cm}{\centering \vspace{2mm}Количество вызовов процедуры \t{send} должно быть не более $K=10 \cdot N$.\\\vspace{2mm}} \\
\hline
4 & 29 & $1 \leq N \leq 32$ & $0 \le R \le 255$ & \parbox{7cm}{\centering \vspace{2mm}Количество вызовов процедуры \t{send} должно быть не более $K=10 \cdot N$.\\\vspace{2mm}} \\
\hline
5 & до 19 баллов & $16 \leq N \leq 64$ &  $0 \le R \le 255$ &  \parbox{7cm}{\centering \vspace{2mm}Количество вызовов процедуры \t{send} должно быть не более $K=15 \cdot N$.

\textbf{Внимание}: количество баллов за эту подзадачу зависит от отношения между длинами закодированного и исходного сообщений.
Для каждого теста с номером $t$ в этой подзадаче пусть величина $P_t=\frac{L_t}{N_t}$ будет равна отношению между длиной закодированной последовательности $L_t$ и длиной исходной последовательности $N_t$. Пусть $P$ будет максимумом среди всех $P_t$. Количество баллов, получаемых в результате оценивания за эту подзадачу, будет определяться
следующими правилами:
\begin{itemize}
\item Если $P \leq 5$, решение получает за эту подзадачу все $19$ баллов.
\item Если $5 < P \leq 6$, решение получает за эту подзадачу $18$ баллов.
\item Если $6 < P \leq 7$, решение получает за эту подзадачу $17$ баллов.
\item Если $7 < P \leq 15$, количество баллов, которое решение получает за эту
подзадачу, будет равно значению выражения $1 + 2 \cdot (15 - P)$, округленному вниз
до ближайшего целого числа.
\item Если $P > 15$ или \textit{хотя бы один из} ваших ответов неверен, решение получает за эту подзадачу $0$ баллов.
\end{itemize}~--- \\\vspace{2mm}} \\
\hline
\end{tabular}
\end{center}


\textbf{Внимание}: Любое правильное решение для подзадач от $1$ до $4$ решает все
предыдущие подзадачи. Однако из-за более высокого ограничения на $K$, правильное
решение для $5$ подзадачи может не решать подзадачи с $1$ по $4$. Есть возможность
решить все подзадачи одним решением.
