Рассмотрим пример, где $N = 3$ и
$M=[10,30,20]$

Процедура \t{encode(N,M)}, используя какой-то неизвестный метод, может закодировать это
сообщение следующей последовательностью чисел: $[7, 3, 2, 70, 15, 20, 3]$. Чтобы сообщить
эту последовательность, она должна вызывать процедуру \t{send} в следующей последовательности:

\t{send(7)}

\t{send(3)}

\t{send(2)}

\t{send(70)}

\t{send(15)}

\t{send(20)}

\t{send(3)}

Предположим, что после того как все попугаи достигли пункта назначения, был получен
следующий список чисел: $[3, 20, 70, 15, 2, 3, 7]$. Процедура \t{decode} будет вызвана с $N=3$, $L=7$ и $X=[3, 20, 70, 15, 2, 3, 7]$

Процедура \t{decode} олжна восстановить исходное сообщение, то есть $[10, 30, 20]$. Она
сообщит результат, вызывая процедуру \t{output} в следующей последовательности:

\t{output(10)}

\t{output(30)}

\t{output(20)}

Примечания о ограничениях:
\begin{itemize}
\item Система оценивания: В реальной системе оценивания ваше решение будет
выполнено дважды, поэтому все ваши глобальные переменные будут очищены между выполнением \t{encode} и \t{decode}. Первый запуск вызывает только \t{encode}, второй~--- только \t{decode}.
только процедуру \t{decode}.
\item Ограничения: Первый запуск не более $50$ раз запустит процедуру \t{encode}. Второй запуск не более $50$ раз запустит процедуру \t{decode}. 
\item Каждый запуск должен использовать процессорное время и память в рамках ограничений. Они оцениваются независимо.
\end{itemize}