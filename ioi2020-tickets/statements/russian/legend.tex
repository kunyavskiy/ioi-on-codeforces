Ринго приехал в Сингапур на карнавал. У него есть несколько призовых билетов в сумке, которые он хочет использовать для участия в розыгрыше призов в киоске. Каждый билет покрашен в один из $n$ цветов. На каждом билете написано целое неотрицательное число. Некоторые написанные на разных билетах числа могут совпадать. Из-за особенностей правил карнавала гарантируется, что $n$ \t{четное}. 

У Ринго в его сумке есть $m$ билетов каждого цвета, таким образом, всего у него есть $n \cdot m$ билетов. На $j$-м билете цвета $i$ написано число $x[i][j]$ ($0 \leq i \leq n-1$ и $0 \leq j \leq m-1$).

Розыгрыш призов проходит в $k$ раундов, пронумерованных от $0$ до $k-1$. Каждый раунд проходит следующим способом:
\begin{itemize}
\item Ринго выбирает $n$ билетов из сумки, по одному билету каждого цвета, и отдает их ведущему игры.
\item Ведущий записывает числа $a[0], a[1] \ldots a[n-1]$, написанные на этих билетах. Порядок этих $n$ чисел не имеет значения.
\item Ведущий получает специальную карточку из лототрона и записывает число $b$, написанное на этой карточке.
\item Ведущий вычисляет модуль разности между $a[i]$ и $b$ для всех $i$ от $0$ до $n-1$. Пусть $S$~--- сумма этих модулей.
\item В этом раунде ведущий выдает Ринго приз в размере $S$.
\item Использованные билеты выбрасываются и не могут быть использованы в следующих раундах.
\end{itemize}

Билеты, оставшиеся в сумке Ринго после $k$ раундов, выбрасываются. 

Присмотревшись внимательно, Ринго понял, что игра~--- нечестная! Внутри лототрона на самом деле находится принтер. В каждом раунде ведущий находит такое число $b$, чтобы приз в этом раунде оказался минимально возможным. Это значение выбирается ведущим и печатается на специальной карте для этого раунда. 

Зная эту информацию, Ринго хочет распределить билеты по раундам. А именно, он хочет выбрать множество билетов для каждого раунда так, чтобы максимизировать суммарный размер полученных призов.

\textbf{Детали реализации}

Вы должны реализовать следующую функцию:

\begin{itemize}
\item \t{int64 find\_maximum(int k, int[][] x)}
\begin{itemize}

\item $k$: количество раундов.
\item $x$: массив размера $n\times m$, описывающий числа на билетах. Билеты каждого цвета отсортированы по неубыванию. 
\item Функция будет вызвана ровно один раз.
\item Функция должна ровно один раз вызвать функцию  \t{allocate\_tickets} (смотрите ниже), описывающую $k$ множеств билетов, по одному множеству на каждый раунд. Этот способ выбрать билеты должен максимизировать суммарный размер призов. 
\item Функция должна вернуть максимальный суммарный размер полученных призов. 
\end{itemize}
\end{itemize}

Функция \t{allocate\_tickets} определена следующим образом:

\begin{itemize}
\item \t{void allocate\_tickets(int[][] s)}
\begin{itemize}
\item $s$: массив размера $n\times m$. Значение $s[i][j]$ должно быть равно $r$, если билет $j$ цвета $i$ используется в раунде $r$, или же $-1$, если билет не используется.
\item Для всех $0 \leq i \leq n - 1$ среди чисел $s[i][0], s[i][1],\ldots, s[i][m - 1]$ каждое значение $0,1,2,\ldots,k-1$ должно встречаться ровно один раз, все остальные элементы должны быть равны $-1$.
\item Если существует несколько способов выбрать билеты, максимизирующих значение суммарного размера призов, можно выбрать любой из них. 
\end{itemize}
\end{itemize}

