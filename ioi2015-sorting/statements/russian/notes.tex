\textbf{Пример 1}

Предположим что:
\begin{itemize}
\item Начальная последовательность: $S=4,3,2,1,0$.
\item Ермек собирается сделать шесть обменов ($M=6$).
\item Последовательности $X$ и $Y$, описывающие позиции, которые Ермек собирается выбрать,
следующие: $X=0,1,2,3,0,1$ и $Y=1,2,3,4,1,2$. То есть, пары позиций, которые
Ермек планирует выбрать, будут $(0,1), (1,2), (2,3), (3,4), (0,1)$ и $(1,2)$.
\end{itemize}

В этом случае Айжан может преобразовать последовательность $S$ к виду $0,1,2,3,4$ за три
этапа. Она может сделать это, выбрав позиции $(0,4), (1,3)$, и затем $(3,4)$.
Следующая таблица отражает, как Ермек и Айжан изменяют последовательность $S$.
\begin{center}
\begin{tabular}{|c|c|c|c|}
\hline
Этап & Игрок & Пара позиций в обмене & Последовательность\\
\hline
начало & & & 4,3,2,1,0\\
\hline
0 & Ермек & (0,1) & 3,4,2,1,0\\ 
\hline
0 & Айжан & (0,4) & 0,4,2,1,3\\
\hline
1 & Ермек & (1,2) & 0,2,4,1,3\\
\hline
1 & Айжан & (1,3) & 0,1,4,2,3\\
\hline
2 & Ермек & (2,3) & 0,1,2,4,3\\
\hline
2 & Айжан & (3,4) & 0,1,2,3,4\\
\hline
\end{tabular}
\end{center}
\textbf{Пример 2}

Предположим, что:
\begin{itemize}
\item Начальная последовательность: $S = 3,0,4,2,1$.
\item Ермек собирается сделать пять обменов $M = 5$.
\item Пары позиций, которые Ермек планирует выбрать: $(1,1), (4,0), (2,3), (1,4)$ и $(0,4)$.
\end{itemize}

В этом случае Айжан может отсортировать последовательность $S$ за три этапа, например,
выбрав пары позиций $(1,4)$, $(4,2)$ и затем $(2,2)$. Следующая таблица отражает, как Ермек и
Айжан изменяют последовательность $S$.
\begin{center}
\begin{tabular}{|c|c|c|c|}
\hline
Этап & Игрок & Пара позиций в обмене &  Последовательность\\
\hline
начало & & & 3,0,4,2,1 \\
\hline
0 & Ермек & (1,1) & 3,0,4,2,1 \\
\hline
0 & Айжан & (1,4) & 3,1,4,2,0 \\
\hline
1 & Ермек & (4,0) & 0,1,4,2,3 \\
\hline
1 & Айжан & (4,2) & 0,1,3,2,4 \\
\hline
2 & Ермек & (2,3) & 0,1,2,3,4 \\
\hline
2 & Айжан & (2,2) & 0,1,2,3,4\\
\hline
\end{tabular}
\end{center}