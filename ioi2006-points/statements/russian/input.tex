СТРОКА 1: Содержит целое число $g$ ($3 \le g \le 50\,000$)~--- количество зеленых точек.

СЛЕДУЮЩИЕ $g$ СТРОК: Каждая строка содержит два целых числа~--- координаты $x_i$ и $y_i$  каждой из $g$ зеленых точек, начиная с точки с номером $1$ и заканчивая точкой с номером $g$. Эти два числа разделены пробелами.

СТРОКА $g + 2$: Содержит целое число $r$ ($3 \le r \le 50\,000$)~--- количество красных точек.

СЛЕДУЮЩИЕ $r$ СТРОК: Каждая строка содержит два целых числа~--- координаты $x_i$ и $y_i$  каждой из $r$ красных точек, начиная с точки с номером $1$ и заканчивая точкой с номером $r$. Эти два числа разделены пробелами.

Точки расположены в соотвествии с условии задачи для некоторого $s$ ($0 < s \le 200\,000\,000$).