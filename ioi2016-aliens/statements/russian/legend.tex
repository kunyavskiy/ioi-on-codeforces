Спутник недавно обнаружил внеземную цивилизацию на далекой планете. Со
спутника получена фотография низкого качества квадратной области на
поверхности этой планеты. Фотография показывает много признаков разумной
жизни. Эксперты определили $n$ интересных точек на этой фотографии. Точки
пронумерованы от $0$ до $n - 1$. Теперь требуется получить новые фотографии в
более хорошем качестве, содержащие все $n$ интересных точек.


Полученная со спутника фотография низкого качества разделяется на сетку,
состоящую из $m$ на $m$ единичных клеток. Строки и столбцы сетки
последовательно пронумерованы от $0$ до $m - 1$ (сверху вниз и слева направо,
соответственно). Для клетки в ряду $s$ и в столбце $t$ используется обозначение $(s, t)$. Точка с номером $i$ содержится внутри клетки $(r_i, c_i)$. Каждая клетка
может содержать любое количество интересных точек.


Спутник находится на стабильной орбите, проходящей ровно над главной
диагональю сетки. Главной диагональю называется диагональ, которая
соединяет левый верхний и правый нижний углы сетки. Спутник может сделать
фотографию в хорошем качестве любой области, удовлетворяющей следующим
условиям:

\begin{itemize}
  \item область должна иметь форму квадрата,
  \item два противоположных угла квадрата лежат на главной диагонали сетки,
  \item каждая клетка сетки находится целиком внутри или целиком снаружи фотографируемой области.
\end{itemize}


Спутник может сделать не более $k$ фотографий в хорошем качестве.


После того как спутник сделал все фотографии, он передает фотографии каждой
сфотографированной клетки в хорошем качестве, не зависимо от того, содержит
ли клетка интересные точки. Каждая сфотографированная клетка будет
передана ровно один раз, даже если она была сфотографирована несколько раз.


Таким образом, необходимо выбрать не более $k$ квадратных областей, которые
будут сфотографированы, добившись, чтобы:

\begin{itemize}
  \item каждая интересная точка была сфотографирована хотя бы один раз, и
  \item количество клеток, которые будут сфотографированы хотя бы один раз, было минимально.
\end{itemize}


Требуется найти минимальное общее количество клеток которые были
сфотографированы хотя бы один раз.

\textbf{Детали реализации}

Требуется реализовать следующую функцию

\begin{itemize}
\item \texttt{ int64 take\_photos(int n, int m, int k, int[] r, int[] c) }
\begin{itemize}

\item \texttt{n}: количество интересных точек,
\item \texttt{m}: количество строк (а также столбцов) в сетке,
\item \texttt{k}: наибольшее количество фотографий, которые спутник может сделать,
\item \texttt{r} и \texttt{c}: два массива длины $n$ описывающие координаты клетки, содержащей интересные точки. Для $0 \le i \le n - 1$, $i$-я интересная точка содержится в клетке $(r[i], c[i])$,
\item функция должна возвращать минимальное количество клеток, которые будут сфотографированы хотя бы один раз, при условии, что фотографии должны покрыть все интересные точки.
\end{itemize}
\end{itemize}

Используйте приведенный шаблон решения для получения деталей реализации на выбранном вами языке программирования.
