The sample grader first calls \texttt{construct\_network(H, W, K)}.
If \texttt{construct\_network} violates some constraint described in the problem statement, the sample grader prints one of the error messages listed at the end of Implementation details section and exits.


Otherwise, the sample grader produces two outputs.

First, the sample grader prints the output of the robot's program in the following format:
\begin{itemize}
\item line $1+i$ $(0 \leq i)$: output of the last instruction in the robot's program for image $i$ ($1$ or $0$).
\end{itemize}

Second, the sample grader writes a file `log.txt' in the current directory in the following format:
\begin{itemize}
\item line $1+i$ $(0 \leq i)$: $m[i][0], m[i][1], \ldots, m[i][c-1]$
\end{itemize}

The sequence on line $1+i$ describes the values stored in the robot's memory cells after the robot's program is run, given image $i$ as the input.
Specifically, $m[i][j]$ gives the value of cell $j$.
Note that the value of $c$ (the length of the sequence) is equal to $H \cdot W$ plus the number of instructions in the robot's program.
