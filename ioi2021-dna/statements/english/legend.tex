Grace is a biologist working in a bioinformatics firm in Singapore. As part of her job, she analyses the
DNA sequences of various organisms. A DNA sequence is defined as a string consisting of
characters ``A'', ``T'', and ``C''. Note that in this task DNA sequences \textbf{do not contain character ``G''}.

We define a mutation to be an operation on a DNA sequence where two elements of the sequence
are swapped. For example a single mutation can transform ``A\textbf{C}T\textbf{A}'' into ``A\textbf{A}T\textbf{C}'' by swapping the
highlighted characters ``A'' and ``C''.

The mutation distance between two sequences is the minimum number of mutations required to
transform one sequence into the other, or $-1$ if it is not possible to transform one sequence into the
other by using mutations.

Grace is analysing two DNA sequences $a$ and $b$, both consisting of $n$ elements with indices from $0$ to $n - 1$. Your task is to help Grace answer $q$ questions of the form: what is the mutation distance between the substring $a[x\ldots y]$ and the substring $b[x\ldots y]$? Here, a substring $s[x\ldots y]$ of a DNA
sequence $s$ is defined to be a sequence of consecutive characters of $s$, whose indices are $x$ to $y$
inclusive. In other words, $s[x\ldots y]$ is the sequence $s[x]s[x+1]\ldots s[y]$.

\textbf{Implementation details}

You should implement the following procedures:

\begin{itemize}
\item \texttt{void init(string a, string b)}
\begin{itemize}

\item $a$, $b$: strings of length $n$, describing the two DNA sequences to be analysed.
\item This procedure is called exactly once, before any calls to \texttt{get\_distance}.
\end{itemize}

\item \texttt{int get\_distance(int x, int y)}
\begin{itemize}
\item $x$, $y$: starting and ending indices of the substrings to be analysed.
\item The procedure should return the mutation distance between substrings $a[x..y]$ and $b[x..y]$.
\item This procedure is called exactly $q$ times.
\end{itemize}
\end{itemize}