Пловдивская Олимпиада по информатике (POI) проходила согласно таким необычным правилам.
Было $N$ участников и $T$ задач. Каждая задача оценивалась с использованием всего лишь одного теста. Таким образом, для каждого участника и каждой задачи было только две
возможности: либо участник решил задачу, либо не решил её. Не было частичного оценивания решения никакой задачи.

Количество баллов, назначенное каждой задаче, определялось после соревнования и было равно количеству участников, которые не решили задачу. Баллы каждого участника подсчитывались как сумма баллов, назначенных задачам, решённым этим участником.

Филипп участвовал в соревновании, но он запутался в сложных правилах оценивания, и сейчас он, глядя на результаты, не в состоянии определить своё место в финальном протоколе.
Помогите Филиппу написать программу, которая вычислит его баллы и его место в финальном протоколе.

Перед соревнованием участникам присвоили уникальные номера от $1$ до $N$ включительно.
Номер Филиппа обозначим буквой $P$. В финальном протоколе участники перечислены в порядке убывания набранных ими баллов. В случае равенства баллов первыми будут перечислены участники, которые решили больше задач. В случае равенства количества решенных задач участники с одинаковыми результатами будут перечислены в порядке возрастания их номеров.

Напишите программу, которая по заданной информации о том, какие задачи были решены какими участниками, определит количество баллов у Филиппа и его место в финальном протоколе.
