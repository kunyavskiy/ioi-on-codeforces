Первая задача не была решена одним участником, то есть, она оценивается в $1$ балл. Вторая
задача не была решена двумя участниками, она оценивается в $2$ балла. Третья задача не была решена четырьмя участниками, она оценивается в $4$ балла. Таким образом, первый участник наберет $4$ балла. Второй участник (Филипп), а также четвертый и пятый участники наберут по $3$ балла каждый. Третий участник наберет $1$ балл. Участник с номерами $2$, $4$ и $5$ имеют одинаковое количество баллов и решили одно и то же количество задач, поэтому в соответствии со вторым правилом (в этом случае участники располагаются в порядке возрастания их номеров) Филипп окажется перед участниками с номерами 4 и 5. Таким образом, в финальном протоколе Филипп будет на втором месте, после участника с номером $1$.

