Базза и Шазза играют в игру. Доска для этой игры представляет собой прямоугольную таблицу, содержащую $R$ строк с номерами от $0$ до $R - 1$ и $C$ столбцов с номерами от $0$ до $C - 1$. Обозначим за $(P, Q)$ ячейку таблицы на пересечении строки $P$ и столбца $Q$. В каждой ячейке записано неотрицательное целое число. В начале игры во всех ячейках записаны нули.

Игра происходит следующим образом. Каждым ходом Базза может:
\begin{itemize}
\item присвоить новое число ячейке $(P, Q)$;
\item либо задать Шаззе вопрос, чему равен наибольший общий делитель (НОД) всех целых чисел внутри прямоугольника, составленного из клеток, противоположные углы которого~--- клетки $(P, Q)$ и $(U, V)$, включительно.
\end{itemize}

Базза делает не более $(N_U + N_Q)$ ходов ($N_U$~--- количество присваиваний чисел ячейкам, а $N_Q$~--- количество вопросов), а потом устаёт и идёт играть в крикет.

Ваша задача~--- определить правильные ответы на вопросы.

Ваше решение должно содержать функции \t{init()} и \t{update()} и функцию \t{calculate()}.

Ваша функция \t{init()}:

\t{void init(int R, int C);}

Ваше решение должно реализовывать эту функцию.

Эта функция задаёт вам размер таблицы и позволяет инициализировать любые глобальные переменные и структуры данных. Она будет вызвана только один раз перед какими-­либо другими вызовами функций \t{update()} или \t{calculate()}.

Параметры:
\begin{itemize}
\item $R$: количество строк.
\item $C$: количество столбцов.
\end{itemize}

Ваша функция \t{update()}:

\t{void update(int P, int Q, long long K);}

Ваше решение должно реализовывать эту функцию.

Эта функция будет вызываться, когда Базза присваивает число какой-­либо ячейке.

Параметры:
\begin{itemize}
\item $P$: строка, в которой находится ячейка $( 0 \leq P \leq R - 1 )$;
\item $Q$: столбец, в котором находится ячейка $( 0 \leq Q \leq C - 1 )$;
\item $K$: новое целое число для этой ячейки $( 0 \leq K \leq 10^{18} )$. Оно может не отличаться от предыдущего числа в этой ячейке.
\end{itemize}

    
Ваша функция \t{calculate()}:

\t{long long calculate(int P, int Q, int U, int V);}

Ваше решение должно реализовывать эту функцию.

Эта функция должна вычислять наибольший общий делитель всех чисел в прямоугольнике с противоположными углами $(P, Q)$ и $(U, V)$, включительно, то есть ячейки $(P, Q)$ и $(U, V)$ включаются в прямоугольник.

Если все числа в этом прямоугольнике~--- нули, эта функция должна возвращать ноль.

Параметры:
\begin{itemize}
\item $P$: строка, в которой находится верхняя левая ячейка прямоугольника $( 0 \leq P \leq R - 1)$.
\item $Q$: столбец, в котором находится верхняя левая ячейка прямоугольника $( 0 \leq Q \leq C - 1)$.
\item $U$: строка, в которой находится нижняя правая ячейка прямоугольника $( P \leq U \leq C - 1)$.
\item $V$: столбец, в котором находится нижняя правая ячейка прямоугольника $( Q \leq V \leq C - 1)$.
\item \textit{возвращаемое значение}: значение НОД всех чисел в данном прямоугольнике или $0$, если все числа в этом прямоугольнике равны нулю.
\end{itemize}
