\textbf{Example 1}

\texttt{solve\_puzzle(``..........", [3, 4])}

These are all possible valid solutions of the puzzle:
\begin{itemize}
\item ``\texttt{XXX\_XXXX\_\_}",
\item ``\texttt{XXX\_\_XXXX\_}",
\item ``\texttt{XXX\_\_\_XXXX}",
\item ``\texttt{\_XXX\_XXXX\_}",
\item ``\texttt{\_XXX\_\_XXXX}",
\item ``\texttt{\_\_XXX\_XXXX}".
\end{itemize}

One can observe that the cells with (0-based) indices 2, 6, and 7 are black in each valid solution. Each of
the other cells can be, but does not have to be black. Hence, the correct answer is ``\texttt{??X???XX??}".

\textbf{Example 2}

\texttt{solve\_puzzle(``........", [3, 4])}

In this example the entire solution is uniquely determined and the correct answer is ``\texttt{XXX\_XXXX}".

\textbf{Example 3}

\texttt{ solve\_puzzle(``...\_.\_....", [3]) }

In this example we can deduce that cell 4 must be white as well~--- there is no way to fit three  consecutive black
cells between the white cells at indices 3 and 5. Hence, the correct answer is ``\texttt{???\_\_\_????}".

\textbf{Example 4}

\texttt{ solve\_puzzle(``.X........", [3])}

There are only two valid solutions that match the above description:

\begin{itemize}
\item ``\texttt{XXX\_\_\_\_\_\_\_}",
\item ``\texttt{\_XXX\_\_\_\_\_\_}".
\end{itemize}

Thus, the correct answer is ``\texttt{?XX?\_\_\_\_\_\_}".

