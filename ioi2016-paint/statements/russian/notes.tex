\textbf{Пример 1}

\texttt{solve\_puzzle("..........", [3, 4])}

Возможны несколько правильных решений головоломки:
\begin{itemize}
\item "\texttt{XXX\_XXXX\_\_}",
\item "\texttt{XXX\_\_XXXX\_}",
\item "\texttt{XXX\_\_\_XXXX}",
\item "\texttt{\_XXX\_XXXX\_}",
\item "\texttt{\_XXX\_\_XXXX}",
\item "\texttt{\_\_XXX\_XXXX}".
\end{itemize}

Можно заметить, что ячейки с индексами 2, 6 и 7 (нумерация начинается с 0) покрашены в черный во всех правильных решениях. Все остальные ячейки могут быть покрашены как в черный, так и в белый цвет. Следовательно, правильный ответ "\texttt{??X???XX??}".

\textbf{Пример 2}

\texttt{solve\_puzzle("........", [3, 4])}

В этом примере требуемое решение определяется однозначно, поэтому правильный ответ "\texttt{XXX\_XXXX}".

\textbf{Пример 3}

\texttt{solve\_puzzle("...\_.\_....", [3])}

В этом примере мы можем определить, что четвертая ячейка будет белая, так как невозможно разместить последовательный блок из трех черных клеток между белыми ячейками с индексами 3 и 5. Следовательно, правильный ответ "\texttt{???\_\_\_????}".

\textbf{Пример 4}

\texttt{solve\_puzzle(".X........", [3])}

В этом случае существует два правильных решения:

\begin{itemize}
\item "\texttt{XXX\_\_\_\_\_\_\_}",
\item "\texttt{\_XXX\_\_\_\_\_\_}".
\end{itemize}

Таким образом, правильный ответ "\texttt{?XX?\_\_\_\_\_\_}".
