Ваша программа посылает команды инопланетному устройству с использованием
стандартного потока вывода и получает ответы из стандартного потока ввода.
Каждая строка с командой должна завершаться одним переводом строки.
\begin{itemize}
	\item
	В начале работы вашей программы вы должны прочитать три целых числа
	$N$, $X_0$ и $Y_0$, разделённые одиночными пробелами,
	из стандартного потока ввода.
	Число $N$ ($15 \le N \le 2\,000\,000\,000$) "--- это размер луга, а $(X_0, Y_0)$ "---
	это координаты одной ячейки со скошенной травой.
	\item
	Чтобы проверить с использованием инопланетного устройства,
	скошена ли трава в ячейке $(X, Y)$, необходимо вывести
	в стандартный поток вывода строку в формате \texttt{examine~X~Y}.
	Если координаты $(X, Y)$ не находятся внутри луга (не выполены
	условия $1 \le X \le N$ и $1 \le Y \le N$), или вы используете
	устройство более $300$ раз, ваша программа получает 0 баллов
	на этом тесте.
	\item
	Инопланетное устройство будет отвечать одной строкой \texttt{true},
	если трава в ячейке $(X, Y)$ скошена,
	в противном случае оно будет отвечать строкой \texttt{false}.
	\item
	Когда ваша программа нашла центральную ячейку, она должна вывести
	строку вида \texttt{solution $X_C$ $Y_C$} в стандартный поток вывода,
	где $(X_C, Y_C)$ "--- координаты центральной ячейки.
\end{itemize}
Для того, чтобы правильно общаться с устройством, ваша программа должна
делать операцию \texttt{flush}.
