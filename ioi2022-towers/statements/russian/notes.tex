
Рассмотрим следующую последовательность вызовов:

\texttt{init(7, [10, 20, 60, 40, 50, 30, 70])}

\texttt{max\_towers(1, 5, 10)}

Пак Денгклек может взять в аренду радиовышки с номерами $1$, $3$, и $5$.
Пример на рисунке ниже, закрашены радиовышки, которые были взяты в аренду.

\includegraphics{towers-example.png}

Радиовышки с номерами $3$ и $5$ могут взаимодействовать с использованием радиовышки с номером $4$ в качестве промежуточной, поскольку $40 \le 50 - 10$ и $30 \le 50 - 10$.
Радиовышки с номерами $1$ и $3$ могут взаимодействовать с использованием радиовышки с номером $2$ в качестве промежуточной.
Радиовышки с номерами $1$ и $5$ могут взаимодействовать с использованием радиовышки с номером $3$ в качестве промежуточной.
Не существует способа взять в аренду больше $3$ радиовышек, следовательно функция должна вернуть значение $3$.

\texttt{max\_towers(2, 2, 100)}

В этом запросе в отрезок номеров попадает только $1$ радиовышка, поэтому Пак Денгклек может взять в аренду только $1$ радиовышку.
Таким образом функция должна вернуть $1$.

\texttt{max\_towers(0, 6, 17)}

Пак Денгклек может взять в аренду радиовышки с номерами $1$ и $3$.
Радиовышки с номерами $1$ и $3$ могут взаимодействовать с использованием радиовышки с номером $2$ в качестве промежуточной, поскольку $20 \le 60 - 17$ и $40 \le 60 - 17$.
Нет способа взять в аренду больше $2$ радиовышек, поэтому функция должна вернуть $2$.

