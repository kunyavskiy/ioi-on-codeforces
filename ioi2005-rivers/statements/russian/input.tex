В первой строке ввода содержится два целых числа $n$ и $k$ "--- количество
поселений лесорубов и количество новых лесопилок, которые можно построить
($2 \le n \le 100$, $1 \le k \le 50$, $k \le n$).

Следующие $n$ строк описывают деревни; $i$-я из этих строк содержит три
целых числа $w_i$, $v_i$ и $d_i$ "--- количество деревьев, ежегодно срубаемых
жителями $i$-го поселения, номер ближайшего населённого пункта вниз по
течению и расстояние до него ($0 \le w_i \le 10\,000$, $0 \le v_i \le n$,
$1 \le d_i \le 10\,000$).

Поселения пронумерованы целыми числами, начиная с единицы, в том порядке,
в котором они заданы во входных данных; городок обозначается числом $0$.

Гарантируется, что до постройки дополнительных лесопилок ежегодные
расходы на сплав составляли не более $2\,000\,000\,000$ монет.

