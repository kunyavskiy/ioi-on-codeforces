% Origin: IOI 2005, Poland (http://www.ioi2005.pl/), Day 2, riv
% Translation: Ivan Kazmenko

В одном уголке земного шара раскинулась живописная страна, покрытая лесами
и реками. Маленькие речушки сливаются в большие, те "--- в ещё
б\textit{о}льшие, и в конце концов все реки сливаются в одну большую
реку, впадающую в море неподалёку от городка.

А ещё в этой стране есть $n$ поселений лесорубов. Каждое поселение расположено
около какой-то реки. Лесорубы валят лес, а затем сплавляют его по воде
до городка. В городке находится лесопилка.

Чтобы сократить расходы на транспортировку леса по воде, решено было
выбрать $k$ поселений лесорубов и построить дополнительные лесопилки
прямо у этих поселений. Из остальных поселений лес будет сплавляться
вниз по течению до первой же лесопилки, будь то старая лесопилка
в городке или одна из $k$ новых. Лес, обработанный в лесопилке,
сплавлять дальше по воде не требуется.

Ни одна река не разветвляется, так что из любого поселения существует
единственный путь вниз по течению до городка.

Для каждого поселения лесорубов известно, сколько деревьев валят жители
этого поселения за год. Требуется поставить новые лесопилки так, чтобы
ежегодные расходы на сплав леса оказались как можно меньше.
Известно, что транспортировка одного дерева на один километр по любому
отрезку речной сети обходится в одну монету.
Напишите программу, которая по заданному описанию речной сети,
а также числу $k$ вычисляет, каковы будут минимальные ежегодные расходы
на сплав леса в монетах.

