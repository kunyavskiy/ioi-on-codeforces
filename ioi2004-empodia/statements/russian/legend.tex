{\bf Обратите внимание, что тесты существенно доработаны, по сравнению с исходными. Система оценки также изменена. }

Античный математик и философ Пифагор верил, что в природе всё подчинено математическим законам. Сегодняшние биологи изучают свойства биопоследовательностей.
\textit{Биопоследовательность} "--- это последовательность из $M$ элементов,
которая
\begin{itemize}
\item
содержит все целые числа от $0$ до $M - 1$,
\item
начинается с $0$ и заканчивается на $M - 1$,
\item
не содержит соседних элементов $E$ и $E + 1$, идущих в таком порядке.
\end{itemize}
Подпоследовательность, состоящая из нескольких подряд идущих элементов
биопоследовательности, называется \textit{отрезком}.

Отрезок биопоследовательности называется \textit{непрерывным},
если он содержит все целые числа в промежутке от значения первого элемента,
который является минимальным на данном отрезке, до значения
последнего элемента, который является максимальным на данном отрезке.
Непрерывный отрезок называется \textit{барьером}, если он не содержит
более коротких непрерывных отрезков.

К примеру, рассмотрим биопоследовательность $(0,3,5,4,6,2,1,7)$.
Вся последовательность является непрерывным отрезком.
Тем не менее, она содержит другой непрерывный отрезок $(3,5,4,6)$
и поэтому не является барьером.
Непрерывный отрезок $(3,5,4,6)$ не содержит более коротких непрерывных
отрезков, то есть является барьером. 
Более того, данная биопоследовательность содержит ровно один барьер.

Напишите программу, которая по данной биопоследовательности находит
все её барьеры.

