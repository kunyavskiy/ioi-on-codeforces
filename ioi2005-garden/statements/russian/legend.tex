Петя "--- владелец самого красивого сада города Новосадска. Он посадил $n$
роз у себя в саду. Пришло лето, цветы выросли большими и красивыми.
Петя понял, что ему не хватает сил заботиться обо всех розах в своём саду.
Поэтому он решил нанять двух садовников себе в помощь. Каждому из них Петя
собирается выделить прямоугольный участок, за всеми розами внутри которого
садовник и будет ухаживать. Участки должны быть непересекающимися и каждый
должен содержать ровно $k$ роз.

Петя хочет построить забор вокруг каждого участка, но денег у него на это
немного. Помогите ему выбрать такие два участка, что суммарная длина забора,
который будет необходимо построить, будет как можно меньше.

Сад представляет собой прямоугольник $l$ метров в длину и $w$ в ширину.
Он разделён на $l \times w$ квадратов $1 \times 1$ метр. Зафиксируем
кооридинатную систему с осями, параллельными сторонам сада. У каждого
квадрата будут целые координаты $(x, y)$, удовлетворяющие ограничениям
$1 \le x \le l$, $1 \le y \le w$. В каждом квадрате может быть любое
число роз.

Участки, которые нужно выбрать, должны иметь стороны, параллельные осям
координат. Координаты углов должны быть целыми. Если
$1 \le l_1 \le l_2 \le l$ и $1 \le w_1 \le w_2 \le w$, прямоугольный участок
с углами $(l_1, w_1)$, $(l_1, w_2)$, $(l_2, w_2)$, $(l_2, w_1)$:
\begin{itemize}
  \item содержит все квадраты с координатами $(x, y)$: $l_1 \le x \le x_2$ и
    $w_1 \le y \le w_2$ и
  \item имеет периметр $2 \cdot (l_2 - l_1 + 1) + 2 \cdot (w_2 - w_1 + 1)$.
\end{itemize}

Два участка должны быть непересекающимися, то есть у них не должно быть ни
одного общего квадрата. Если у них есть общая сторона, либо часть стороны,
эта часть всё равно должна быть отделена двумя различными заборами.

Напишите программу, которая найдёт пару непересекающихся участков, каждый
из которых содержит ровно $k$ роз, с минимальной суммой периметров.

