В национальном персидском эпосе Шахнаме есть легенда, повествующая о легендарном
персидском герое Зале, который влюблён в Рудабу, принцессу Кабула. Когда Заль попросил
руки Рудабы, её отец решил дать жениху задание.

В Персии есть $n$ городов, пронумерованных от $0$ до $n - 1$, которые соединены $m$
двусторонними дорогами, пронумерованными от $0$ до $m - 1$. Каждая дорога соединяет пару
различных городов. Каждую пару городов соединяет не более одной дороги. Некоторые
дороги называются \texttt{шахскими} и используются для передвижения членов шахской семьи.
Задача Заля состоит в том, чтобы определить, какие дороги являются шахскими.

У Заля есть карта, на которой отмечены все города и дороги в Персии. Он не знает, какие
дороги являются шахскими, но он может попросить помощи у Симург, легендарной птицы,
покровительствующей Залю. Однако, Симург не хочет просто так сообщить Залю, какие
дороги являются шахскими. Вместо этого Симург сообщила, что набор из шахских дорог
является \texttt{золотым}. Набор дорог является золотым тогда и только тогда, когда:
\begin{itemize}
\item он состоит \texttt{ровно} из $n - 1$ дорог.
\item для каждой пары городов возможно добраться от одного из них до другого, используя
для перемещения только дороги из набора.
\end{itemize}

Теперь Заль может задавать Симург вопросы. Каждый вопрос устроен следующим образом:
\begin{enumerate}
\item Заль выбирает \texttt{золотой} набор дорог
\item Симург сообщает Залю, сколько из выбранных им дорог являются шахскими.
\end{enumerate}

Ваша программа должна помочь Залю найти, какие дороги являются шахскими, задав Симург
не более $q$ вопросов. Проверяющий модуль будет играть роль Симург.

\textbf{Детали реализации}

Вы должны реализовать следующую функцию (метод):

\begin{itemize}
\item \texttt{int[] find\_roads(int n, int[] u, int[] v)}
\begin{itemize}
\item $n$: количество городов,
\item $u$ и $v$: массивы длины $m$. Для каждого $0 \leq i \leq m - 1$ верно, что $u[i]$ и $v[i]$ это города, соединённые дорогой  $i$.
\item Функция должна вернуть массив длины $n - 1$, содержащий номера шахских дорог (в произвольном порядке).
\end{itemize}
\end{itemize}

Ваше решение может произвести не более $q$ вызовов следующей функции проверяющего
модуля:

\begin{itemize}
\item \texttt{int count\_common\_roads(int[] r)}
\begin{itemize}
\item $r$: массивы длины $n - 1$, содержащий номера дорог в золотом наборе дорог (в
произвольном порядке).
\item Функция возвращает число шахских дорог в массиве $r$.
\end{itemize}
\end{itemize}
