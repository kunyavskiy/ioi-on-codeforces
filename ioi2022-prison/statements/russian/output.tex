Грейдер вызывает функцию \texttt{devise\_strategy(N)}.
Значение $x$ определяется как длина результирующего массива, минус один.
После этого если грейдер обнаруживает, что результат функции \texttt{devise\_strategy} не соответствует требованиям из условия, то он печатает одно из следующих сообщений об ошибке и завершает работу:

\begin{itemize}
    \item \texttt{s is an empty array}: $s$ является пустым массивом (и не является корректной стратегией).
\item \texttt{s[i] contains incorrect length}: существует такой индекс $i$ ($0 \le i \le x$), для которого длина массива $s[i]$ не равна $N + 1$.
\item \texttt{First element of s[i] is non-binary}: существует такой индекс $i$ ($0 \le i \le x$), для которого $s[i][0]$ не равно $0$ или $1$.
\item \texttt{s[i][j] contains incorrect value}: существуют такие индексы $i, j$ ($0 \le i \le x, 1 \le j \le N$), что $s[i][j]$ не находится в диапазоне от $-2$ до $x$.

\end{itemize}


В противном случае, грейдер выводит два набора выходных данных.

Во-первых, грейдер выводит результат выполнения вашей стратегии на тестовых примерах.
\begin{itemize}
    \item  строка $1 + k$ ($0 \le k$): результат выполнения стратегии для сценария $k$.
В случае, если следуя стратегии результатом является выбор сумки А, грейдер выводит символ `A'.
В случае, если следуя стратегии результатом является выбор сумки Б, грейдер выводит символ `B'.
В случае, если следуя стратегии ни один из заключенных не выберет сумку для финального ответа, грейдер выводит символ `X'.
\end{itemize}

Во-вторых, грейдер создает файл `log.txt' в текущей директории, в который он выводит данные в следующем формате:

\begin{itemize}
    \item строка $1 + k$ ($0 \le k$): $w[k][0] \; w[k][1] \; \ldots$
\end{itemize}


Последовательность в строке $1 + k$ соответствует примеру $k$ и описывает последовательность чисел, записанных на доске во время выполнения стратегии для примера $k$. Более конкретно, $w[k][l]$ обозначает число, записанное $(l+1)$-ым по счету заключенным (в том порядке, в котором они заходили в комнату).
