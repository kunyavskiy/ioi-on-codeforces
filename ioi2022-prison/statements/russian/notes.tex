
Рассмотрим следующий вызов функции:

\texttt{devise\_strategy(3)}

Пусть $v$ обозначает то число, которое увидит заключенный при входе в комнату.
Одна из допустимых стратегий такая:

\begin{itemize}
\item Если $v = 0$ (в том числе в самом начале испытания), загляни в сумку А.
\begin{itemize}
  \item Если в ней $1$ монета, то выбери сумку А для финального ответа.
  \item Если в ней $3$ монеты, то выбери сумку Б для финального ответа.
  \item Если в ней $2$ монеты, то запиши число $1$ на доске (стерев $0$).
   \end{itemize}
\item Если $v = 1$, загляни в сумку Б.
\begin{itemize}
  \item Если в ней $1$ монета, то выбери сумку Б для финального ответа.
  \item Если в ней $3$ монеты, то выбери сумку A для финального ответа.
  \item Если в ней $2$ монеты, то запиши число $0$ на доске (стерев $1$). Обратите внимание, что такая ситуация не может произойти никогда, так как в этом случае обе сумке содержат по $2$ монеты, что невозможно по условию.
  \end{itemize}
 \end{itemize}
 
Для того, чтобы выдать описанную стратегию, функция должна вернуть \texttt{[[0, -1, 1, -2], [1, -2, 0, -1]]}.
Длина возвращаемого массива равна $2$, таким образом, значение $x$ для данного массива определяется как $2 - 1 = 1$.


