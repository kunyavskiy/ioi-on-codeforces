
\begin{center}
\renewcommand{\arraystretch}{1.5}
\begin{tabular}{|c|c|c|}
\hline
Подзадача & Баллы & \parbox{11cm}{\centering \vspace{2mm}Дополнительные ограничения на входные данные\\\vspace{2mm}} \\
\hline
1 & 5 &\parbox{11cm}{\centering \vspace{2mm} $N \le 500$, значение $x$ не должно превосходить $500$\\\vspace{2mm}}\\
\hline
2 & 5 & \parbox{11cm}{\centering \vspace{2mm}$N \le 500$, значение $x$ не должно превосходить $70$\\\vspace{2mm}}\\
\hline
3 & 90 & \parbox{11cm}{\centering \vspace{2mm}Значение $x$ не должно превосходить $60$.\\\vspace{2mm}}\\
\hline
\end{tabular}
\end{center}

Если в каком-либо тесте результат функции \texttt{devise\_strategy} не задает корректную стратегию, вы получите за эту подзадачу $0$ баллов.

В позадаче 3 вы можете получить частичный балл.
Пусть $m$ обозначает максимально значение $x$ среди всех выданных стратегий на тестах данной позадачи. Ваш балл за эту подзадачу будет определяться согласно следующей таблице:
\begin{center}
 \renewcommand{\arraystretch}{1.5}
 \begin{tabular}{|c|c|c|}
\hline
Ограничение         &  Баллы \\ 
\hline
$40 \le m \le 60$ &  $20$ \\
\hline
$26 \le m \le 39$ &  $25 + 1.5 \times (40 - m)$ \\
\hline
$m = 25$          &  $50$ \\
\hline
$m = 24$          &  $55$ \\
\hline
$m = 23$          &  $62$ \\
\hline
$m = 22$          &  $70$ \\
\hline
$m = 21$          &  $80$ \\
\hline
$m \le 20$        &  $90$ \\
\hline
\end{tabular}
\end{center}


