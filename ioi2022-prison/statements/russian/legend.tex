В тюрьме находятся $500$ заключенных. Начальник тюрьмы решил предложить им шанс выбраться на свободу. Он положил в комнату две сумки с деньгами, сумку А и сумку Б. В каждой из сумок находится от $1$ до $N$ монет, включительно. В сумках находится \textbf{различное} число монет. Начальник предлагает им испытание. Задача заключенных состоит в том, чтобы определить сумку, в которой находится меньше монет.

В комнате помимо двух сумок с деньгами присутствует также маркерная доска. В любой момент времени на доске должно быть написано ровно одно целое число. Изначально на доске записано число $0$.

После этого начальник приглашает заключенных в комнату по одному. Каждый из заключенных не знает, кто именно и сколько заключенных посетили комнату до него.
Каждый раз, когда заключенный заходит в комнату, он видит число, написанное на доске. После того, как заключенный прочитал число на доске, он может выбрать сумку А или сумку Б. Затем заключенный \textbf{заглядывает} в выбранную сумку и узнает число монет в ней. После этого заключенный обязан выполнить одно из двух \textbf{действий}:
 \begin{itemize}
\item Заменить число, записанное на доске, на неотрицательное целое число и покинуть комнату. Обратите внимание, что он может как заменить число на другое, так и оставить то же самое число. После того, как заключенный покинул комнату, испытание продолжается (за исключением случая, когда в комнате уже побывали все $500$ заключенных).
\item Сообщить начальнику, в какой сумке находится меньше монет. После этого испытание завершается.
 \end{itemize}
Начальник никогда не приглашает в комнату заключенного, который в ней уже побывал.

Испытание считается пройденным, если один из заключенных успешно определит сумку, в которой находится меньше монет. Испытание считается непройденным, если один из них указал на неправильную сумку, или же все $500$ заключенных побывали в комнате, но никто из них не попытался угадать, в какой сумке меньше монет.

Перед началом испытания заключенные собираются вместе, чтобы определить общую \textbf{стратегию} из трех шагов:
 \begin{itemize}
\item Они выбирают неотрицательное число $x$, обозначающее максимальное число, которое любой из заключенных будет писать на доске.
\item Они выбирают для каждого числа $i$, которое может быть написано на доске ($0 \le i \le x$), в какую сумку должен заглянуть заключенный, если видит на доске число $i$.
\item Они выбирают, что должен сделать заключенный после того, как заглянул в выбранную сумку. Более конкретно, для каждого из чисел $i$, которое может быть записано на доске ($0 \le i \le x$), и для каждого числа монет $j$ ($1 \le j \le N$), увиденного в выбранной сумке, заключенные выбирают
 \begin{itemize}
  \item либо какое число между $0$ и $x$ (включительно) заключенный запишет на доске,
  \item либо какую сумку заключенный должен выбрать для финального ответа как ту, в которой меньше монет.
 \end{itemize}
  \end{itemize}
После прохождения испытания начальник выпустит всех заключенных через $x$ дней.

Ваша задача состоит в том, чтобы помочь заключенным придумать стратегию, гарантирующую прохождение испытания (вне зависимости от числа монет в сумках А и Б). Оценка вашего решения зависит от числа $x$ (смотрите секцию Subtasks).


\textbf{Implementation Details}

Вы должны реализовать функцию

\begin{itemize}
\item \texttt{int[][] devise\_strategy(int N)}
 \begin{itemize}
\item $N$: максимальное число монет в каждой из сумок
\item Эта фукнция должна возвращать массив $s$, состоящий из массивов целых чисел длины $N + 1$, описывающих вашу стратегию.
Значение $x$ определяется как длина массива $s$, минус один.
Для каждого $i$, где $0 \le i \le x$, массив $s[i]$ должен обозначать то действие, которое должен выполнить заключенный, если видит на доске число $i$:
 \begin{enumerate}
  \item Значение $s[i][0]$ должно быть равно $0$, если заключенный должен заглянуть в сумку А, или $1$, если заключенный должен заглянуть в сумку Б.
 \item Пусть $j$ обозначает число монет, которое заключенный обнаружил в выбранной сумке. Заключенный должен сделать следующее:
  \begin{itemize}
    \item Если значение $s[i][j]$ равно $-1$, то заключенный должен выбрать сумку А как сумку для финального ответа, в которой находится меньшее число монет.
    \item Если значние $s[i][j]$ равно $-2$, то заключенный должен выбрать сумку Б как сумку для финального ответа, в которой находится меньшее число монет.
    \item Если значение $s[i][j]$ является неотрицательным, то заключенный должен записать его на доске. Обратите внимание, что $s[i][j]$ не должно превосходить $x$.
     \end{itemize}
     \end{enumerate}
\item Эта фукнция будет вызвана ровно один раз.
 \end{itemize}
  \end{itemize}

