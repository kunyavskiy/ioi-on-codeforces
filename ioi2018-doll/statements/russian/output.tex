В результате исполнения проверяющий модуль создаёт три файла.

Файл out.txt содержит вывод вашей программы в следующем формате.
\begin{itemize}
    \item строка $1$: $S$
    \item строка $2 + i$ ($0 \le i \le M$): $C[i]$
    \item строка $2 + M + j$ ($1 \le j \le S$): $X[j-1]\ Y[j-1]$
\end{itemize}

Далее проверяющий модуль симулирует перемещения шарика. Он выводит
последовательность серийных номеров устройств, в которых побывал шарик, в
файл log.txt.

Наконец, проверяющий модуль печатает на стандартный поток вывода результат
оценки вашего решения.

\begin{itemize}
    \item Если вердикт вашей программы \textbf{Accepted}, выводятся значения $S$ и $P$ в
следующем формате Accepted: $S\ P$.
   \item Если вердикт вашей программы \textbf{Wrong Answer}, выводится сообщение Wrong
Answer: MSG, где MSG~--- одно из:
    \begin{itemize}
        \item \texttt{answered not exactly once}: Процедура answer вызывана более одного
раза.
\item \texttt{wrong array length}: Длина $C$ не равна $M + 1$ или длины $X$ и $Y$ различны.
\item \texttt{over $400000$ switches}: $S$ больше $400\,000$.
\item \textbf{wrong serial number}: : Массив $C$, $X$ или $Y$ содержит одно или несколько
значений, которые меньше $-S$ или больше . $M$.
\item \texttt{over $20\,000\,000$ inversions}: В процессе перемещений шарик не
возвращается в источник после $20\,000\,000$ изменений состояний
переключателей.
\item \texttt{state `Y'}: Хотя бы один переключатель находится в состоянии `Y' после
первого возвращения шарика в источник.
\item \texttt{wrong motion}: Триггеры, посещенные шариком, не образуют
последовательность $A$.
    \end{itemize}
\end{itemize}

Если ваша программа получает вердикт \texttt{Wrong Answer}, файлы out.txt и/или
log.txt могут быть не созданы.





