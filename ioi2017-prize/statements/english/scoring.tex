In some test cases the behavior of the grader is adaptive. This means that in these test cases the grader does not have a fixed sequence of prizes. Instead, the answers given by the grader may depend on the questions asked by your solution. It is guaranteed that the grader answers in such a way that after each answer there is at least one sequence of prizes consistent with all the answers given so far.

\begin{center}
\renewcommand{\arraystretch}{1.5}
\begin{tabular}{|c|c|c|}
\hline
Subtask & Points & Additional Input Constraints\\
\hline
1 &  20 & \parbox{13cm}{\centering \vspace{2mm}There is exactly 1 diamond and $n - 1$ lollipops (hence, $v = 2$).
You can call the procedure \texttt{ask} at most $10\,000$ times \\\vspace{2mm}} \\
\hline
2 & 80 & No additional constraints \\
\hline
\end{tabular}
\end{center}

In subtask 2 you can obtain a partial score. Let $q$ be the maximum number of calls to the procedure \texttt{ask} among all test cases in this subtask. Then, your score for this subtask is calculated according to the following table:

\newcommand{\lt}{\textless}

\begin{center}
\renewcommand{\arraystretch}{1.5}
\begin{tabular}{|c|c|}
\hline
Questions & Score \\
\hline
$10\,000 \lt q$ & $0$ (reported in CMS as `Wrong Answer') \\
\hline
$6000 \lt q \leq 10\,000$ & $70$ \\
\hline
$5000 \lt q \leq 6000$ & $80 - (q-5000)/100$ \\
\hline
$q \leq 5000$ & $80$ \\
\hline
\end{tabular}
\end{center}


