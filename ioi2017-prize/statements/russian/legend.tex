Большой приз~--- известная телевикторина. Вам повезло стать одним из участников
финального раунда. Напротив вас в ряд выставлены $n$ коробок. Коробки пронумерованы
целыми числами от $0$ до $n - 1$ слева направо. В каждой из коробок находится приз, который
невозможно увидеть, не открыв коробку. В телевикторине используются призы $v \geq 2$ различных \texttt{типов}. Типы пронумерованы целыми числами от $1$ до $v$ в порядке \texttt{убывания} ценности призов.

Самый дорогой приз~--- бриллиант (тип $1$). Ровно в одной коробке находится бриллиант.
Самый дешевый приз~--- леденец (тип $v$). Чтобы игра была более захватывающей, количество
более дешевых призов значительно больше, чем количество более дорогих. Строго говоря,
для всех $t$ ($2 \leq t \leq v$) известно следующее: если имеется $k$ призов типа $t - 1$, то призов типа $t$ \texttt{строго} больше $k^2$.

В конце игры вы должны открыть коробку и получить приз, который в ней находится. Ваша
задача состоит в том, чтобы выиграть бриллиант. До выбора коробки, которая будет открыта,
вы можете задать ведущему телевикторины Рамбоду несколько вопросов. Для каждого
вопроса вы выбираете некоторую коробку с номером $i$. В ответ Рамбод предоставит массив $a$, состоящий из двух целых чисел. Эти числа означают следующее:

\begin{itemize}
\item Среди всех коробок, расположенных слева от коробки с номером $i$, ровно $a[0]$ коробок содержат приз дороже, чем приз в коробке с номером $i$.
\item Среди всех коробок, расположенных справа от коробки с номером $i$, ровно $a[1]$ коробок содержат приз дороже, чем приз в коробке с номером $i$.
\end{itemize}

Например, предположим, что $n = 8$. Для вопроса вы выбрали коробку с номером $i = 2$, и в
ответ Рамбод говорит, что $a = [1, 2]$. Этот ответ означает следующее:

\begin{itemize}
\item Ровно одна коробка из коробок с номерами $0$ и $1$ содержит приз дороже, чем коробка с
номером $2$.
\item Ровно две коробки из коробок с номерами $3, 4, \ldots, 7$ содержат призы дороже, чем коробка с номером $2$.
\end{itemize}

Требуется найти коробку, содержащую бриллиант, уложившись в требуемое количество
вопросов.

\textbf{Детали реализации}

Вам следует реализовать следующую функцию (метод):

\begin{itemize}
\item \texttt{int find\_best(int n)}
\begin{itemize}
\item Эта функция вызывается из проверяющего модуля ровно один раз.
\item $n$: количество коробок.
\item Функция должна вернуть номер коробки, в которой находится бриллиант, то есть такое
уникальное целое число $d$ ($0 \leq d \leq n - 1$) что в коробке с номером $d$ содержится приз типа $1$.
\end{itemize}
\end{itemize}

В этой функции разрешено вызывать следующую функцию:

\begin{itemize}
\item \texttt{int[] ask(int i)}
\begin{itemize}
\item $i$: номер коробки, которую вы выбрали для вопроса. Значение $i$ должно быть от $0$ до $n - 1$ включительно.
\item Функция возвращает массив $a$ из двух элементов. Значение $a[0]$ равно количеству более дорогих призов, расположенных в коробках слева от коробки с номером $i$.  Значение $a[1]$ iравно количеству более дорогих призов, расположенных в коробках справа от коробки с номером $i$.
\end{itemize}
\end{itemize}

