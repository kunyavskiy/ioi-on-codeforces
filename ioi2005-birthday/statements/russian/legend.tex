% Origin: IOI 2005, Poland (http://www.ioi2005.pl/), Day 2, bir
% Translation: Yury Petrov

У Пети сегодня день рождения. На праздник пришло $n$ детей, включая Петю.
Их пронумеровали натуральными числами от $1$ до $n$. Вокруг большого круглого
стола расставили $n$ стульев. Дети приходят по одному и рассаживаются вокруг
стола: первый садится на первое место, второй "--- на следующее слева, и так
далее. Наконец, ребёнок под номером $n$ занимает последнее место между
первым и ($n - 1$)-м.

Родители Пети хорошо знают всех детей и знают, что некоторые дети будут
шуметь слишком сильно, если их посадить рядом. Поэтому они решили пересадить
детей особым образом. Нужный порядок можно описать перестановкой
$p_1, p_2, \ldots, p_n$: ребёнок $p_1$ должен будет сесть между $p_n$ и
$p_2$, ребёнок $p_2$ "--- между $p_1$ и $p_3$, $\ldots$,
ребёнок $p_n$ "--- между $p_{n - 1}$ и $p_1$. Заметьте, что ребёнок
$p_1$ может сесть как слева, так и справа от ребёнка $p_n$.

Чтобы пересадить детей нужным образом, родителям нужно будет пересаживать
детей вправо и влево на несколько мест. Для каждого ребёнка им нужно выбрать
направление перемещения (вправо или влево) и расстояние (число мест).
По сигналу дети одновременно встанут со своих мест, перейдут к нужным местам
и займут их.

Пересаживание неизбежно повлечёт за собой беспорядок. Назовём значением
беспорядка максимальное число мест, на которое придётся переместиться
какому-то из детей. Помогите найти такой способ пересадить всех детей, чтобы
значение беспорядка было минимальным.

