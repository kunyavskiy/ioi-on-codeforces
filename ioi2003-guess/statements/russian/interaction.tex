Первая строка входного файла содержит два целых числа $N$ и $P$, разделенных пробелом.

Каждая из последующих $N$ строк описывает признаки коровы, используя $P$ букв, разделенных пробелами. Первая буква каждой строки~--- значение признака 1 и так далее. Вторая строка во входном файле описывает корову с номером 1, третья строка~--- корову с номером 2 и т.д.

Ваша программа задает вопрос о корове посредством записи в стандартный вывод
строки следующего вида: буква 'Q', номер признака, значения признака (одно или
более), разделенные пробелами. Например, строка "Q 1 Z Y" соответствует
вопросу: "Имеет ли первый признак коровы значение 'Z' или 'Y'?". Признак может быть целым числом в пределах от 1 до $P$. Значения признака должны быть только
'X', 'Y' или 'Z' и не должны повторяться в одной строке. 

После задания вопроса ваша программа должна читать одну строку, содержащую
одно из целых чисел~--- $0$ или $1$. Число $1$ обозначает, что корова обладает одним из
указанных значений признака. Число $0$ означает, что ни одним из указанных
значений признака корова не обладает.

Последняя строка вывода программы должна содержать букву 'C', пробел и одно
число (номер коровы). 
