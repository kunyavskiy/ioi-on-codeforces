Мариам~--- инженер-электрик. Она разрабатывает схему электропроводки для коммуникационной вышки. На вышке находятся несколько точек подключения, расположенные на различной высоте. Провод может быть использован для соединения любых двух точек подключения. К каждой точке подключения может быть подсоединено любое количество проводов. Каждая точка подключения может быть одного из двух типов: красная или синяя.

По условию задачи вышка представлена прямой линией, а точки подключения~--- синими и красными точками, расположенными в целых неотрицательных координатах на этой прямой. Длина провода, соединяющего две точки подключения, равна расстоянию между этими точками.

Необходимо помочь Мариам найти схему электропроводки, удовлетворяющую условиям:

\begin{itemize}
\item Каждая точка подключения соединена хотя бы одним проводом с точкой подключения
другого цвета.
\item Суммарная длина проводов минимальна.
\end{itemize}

\textbf{Детали реализации}

Вам следует реализовать одну функцию (метод):

\begin{itemize}
\item \texttt{int64 min\_total\_length(int[] r, int[] b)}
\begin{itemize}
\item $r$: массив длины $n$, содержащий координаты красных точек подключения (в порядке
возрастания координат).
\item $b$: массив длины $m$, содержащий координаты синих точек подключения (в порядке
возрастания координат).
\item Функция должна вернуть минимальную суммарную длину проводов среди всех
корректных схем электропроводки
\item Возвращаемое значение имеет тип `int64'.
\end{itemize}
\end{itemize}


