Вы играете в видеоигру. Игровой контроллер содержит $4$ кнопки: A, B, X и Y. В этой игре вы можете получать монеты за комбо-действия. Вы можете совершить комбо-действие, нажимая кнопки в определенной последовательности.

В этой игре есть секретная последовательность нажатий кнопок, которую можно представить строкой $S$, каждый символ которой --- один из $4$ символов кнопок. Строка $S$ неизвестна, но известна ее длина $N$.

Также известно, что первый символ строки $S$ никогда в ней больше не встречается. Например, $S$ может быть равна \t{"ABXYY"} или \t{"XYYAA"}, но не может быть равна \t{"AAAAA"} или \t{"BXYBX"}.

Для совершения комбо-действия вы можете последовательно нажать не более $4 \cdot N$ кнопок. Пусть строка $p$ представляет последовательность нажатых кнопок. Количество монет, которое вы получите за это действие, вычисляется как длина наибольшего префикса строки $S$, который также является подстрокой $p$. Подстрока строки $t$ --- это непрерывная (возможно пустая) последовательность символов строки $t$. Префикс строки $t$ --- это пустая или содержащая первый символ подстрока строки $t$.

Например, если $S$ равна \t{"ABXYY"} и $p$ равна \t{"XXYYABYABXAY"}, то будет получено $3$ монеты, потому что \t{"ABX"} это наибольший префикс $S$, который также является подстрокой $p$.

Требуется определить секретную строку $S$, используя комбо-действия.

\bf{Детали реализации}

Вам следует реализовать одну функцию:

\t{string guess_sequence(int N)}

\begin{itemize}
\item $N$: длина строки $S$.
\item Функция вызывается ровно один раз для каждого теста.
\item Функция должна вернуть строку $S$.
\end{itemize}

Ваша программа может вызывать следующую функцию:

\t{int press(string p)}

\begin{itemize}
\item $p$: последовательность нажатых кнопок.
\item $p$ должна быть строкой длины от $0$ до $4 \cdot N$ символов, включительно. Каждый символ строки $p$ должен быть одним из A, B, X или Y.
\item Вы можете вызывать эту функцию не более $8000$ раз для каждого теста.
\item Функция возвращает количество монет, которое будет получено при нажатии последовательности кнопок, представленной строкой $p$.
\end{itemize}

Если какие-либо из вышеописанных условий не выполняются, ваша программа получит вердикт \bf{Wrong Answer}. В противном случае, ваша программа получит вердикт \bf{Accepted}, и ваш балл определяется количеством вызовов функции \t{press} (смотрите раздел "Подзадачи").

\bf{Пример}

Пусть $S$ равна \t{"ABXYY"}. Проверяющий модуль (grader) вызывает функцию \t{guess_sequence(5)}. Пример взаимодействия описан ниже.

\begin{tabular}{|l|l|}\hline
\bf{Вызов}&\bf{Результат}\\\hline
\t{press("XXYYABYABXAY")}&$3$\\\hline
\t{press("ABXYY")}&$5$\\\hline
\t{press("ABXYYABXYY")}&$5$\\\hline
\t{press("")}&$0$\\\hline
\t{press("X")}&$0$\\\hline
\t{press("BXYY)}&$0$\\\hline
\t{press("YYXBA")}&$1$\\\hline
\t{press("AY")}&$1$\\\hline
\end{tabular}



Для первого вызова функции \t{press}, \t{"ABX"} встречается в \t{"XXYYABYABXAY"} как подстрока, а \t{"ABXY"} не встречается, поэтому возвращаемое значение $3$.

Для третьего вызова функции \t{press}, \t{"ABXYY"} целиком встречается в \t{"ABXYYABXYY"} как подстрока, поэтому возвращаемое значение $5$.

Для шестого вызова функции \t{press}, никакой префикс \t{"ABXYY"}, кроме пустой строки, в \t{"BXYY"} не встречается как подстрока, поэтому возвращаемое значение $0$.

Наконец, \t{guess_sequence(5)} должна вернуть строку \t{"ABXYY"}.

Файл \t{sample-01-in.txt} в прикрепленном архиве соответствует этому примеру.

\bf{Ограничения}

\begin{itemize}
\item $1 \le N \le 2000$.
\item Каждый символ строки является одним из A, B, X или Y.
\item Первый символ строки $S$ никогда больше не встречается в $S$.
\end{itemize}

В этой задаче проверяющий модуль НЕ ЯВЛЯЕТСЯ адаптивным. Это означает, что строка $S$ зафиксирована на момент запуска проверяющего модуля и не зависит от выполненных запросов в вашем решении. 

\bf{Пример проверяющего модуля}

 Пример проверяющего модуля читает входные данные в следующем формате: 

\begin{itemize}
\item Строка 1: $S$
\end{itemize}

Если ваша программа получает вердикт \bf{Accepted}, проверяющий модуль печатает \t{"Accepted: q"}, где $q$ равно количеству вызовов функции \t{press}. 
Если ваша программа получает вердикт \bf{Wrong Answer}, проверяющий модуль печатает \t{"Wrong Answer: MSG"}. Значение $MSG$ может быть: 

\begin{itemize}
\item invalid press: Значение $p$, переданное в функцию \t{press}, ошибочно. А именно, длина строки $p$ не в промужетке от $0$ до $4*N$, включительно, или какой-то из символов строки $p$не является ни одним из A, B, X или Y. 
\item too many moves: Функция \t{press} вызвана более $8\,000$ раз. 
\item wrong guess: Возвращаемое функцией \t{guess_sequence} значение не совпадает со строкой $S$. 
\end{itemize}