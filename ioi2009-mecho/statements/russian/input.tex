Ваша программа должна читать со стандартного потока ввода такие данные:
\begin{itemize}
\item Первая строка содержит целые числа $N$ ($1 \le N \le 800$) и $S$ ($1 \le S \le 1\,000$), разделенные пробелом.
\item Последующие $N$ строк задают карту леса. Каждая из этих строк содержит $N$ символов, каждый
символ задает одну ячейку на сетке. Возможные символы и их значения описаны ниже:
\begin{itemize}
\item \t{T} обозначает ячейку с деревом
\item \t{G} обозначает ячейку с травой
\item \t{M} обозначает начальное расположение Миши и горшочка меда в ячейке с травой
\item \t{D} обозначает ячейку где расположен Мишин дом, в который Миша может попасть, а пчелы не
могут
\item \t{H} обозначает ячейку с ульем
\end{itemize}
\end{itemize}

Гарантируется, что карта леса содержит ровно одну букву \t{M}, ровно одну букву
\t{D} и, по крайней мере, одну букву \t{H}. Также гарантируется, что существует последовательность смежных ячеек \t{G}, которые соединяют ячейку с начальным расположением Миши и ячейку, где расположен Мишин дом, так же как и последовательность смежных ячеек \t{G}, которые соединяют хотя бы одну из ячеек с ульем с ячейкой с горшочком (то есть, с ячейкой с Мишиным начальным расположением). Последовательности могут быть и с длиной, равной нулю, в случае, если ячейка с Мишиным домом или ячейка с ульем являются смежными с ячейкой с начальным расположением Миши. Также заметьте, что пчелы не могут распространяться через ячейку с Мишиным домом. Для пчел она~--- как ячейка с деревом.