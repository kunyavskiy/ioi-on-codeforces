Магистральная сеть Сингапура (SIB) состоит из $n$ станций, каждой из которых присвоен \t{индекс} от $0$ до $n-1$. В сети также присутствуют $n-1$ двусторонних соединений, пронумерованных от $0$ до $n-2$, каждое из которых соединяет две различных станции.
Две станции, соединенные напрямую, называются соседями.

Путем от станции $x$ до станции $y$ называется последовательность попарно различных станций $a_0,a_1,\cdots,a_p$ такая, что $a_0=x$, $a_p=y$, а любые две последовательные станции являются соседями. Гарантируется, что существует \t{ровно один} путь от любой станции $x$ до любой другой станции $y$.

Любая из станций $x$ может создать пакет (фрагмент данных) и отправить его любой станции $y$, которая называется \t{пунктом назначения} пакета. Этот пакет должен быть передан от станции $x$ до станции $y$ следующим образом.
Пусть пакет в текущий момент находится на станции $z$, а пунктом назначения этого пакета является станция $y$ ($z \neq y$).
В этом случае станция $z$
\begin{enumerate}
\item выполняет \t{процедуру маршрутизации}, которая определяет соседа $z$, который лежит на единственном пути от $z$ до $y$, и
\item передает пакет этому соседу.
\end{enumerate}

Однако, станции имеют ограниченный размер памяти и не содержат весь список соединений в SIB для использования в процедуре маршрутизации.

Ваша задача состоит в том, чтобы составить схему маршрутизации SIB, которая состоит из двух функций.
\begin{itemize}
\item Первая функция получает на вход число $n$, список соединений SIB и число $k \geq n-1$. Она присваивает каждой из станций \t{уникальный идентификатор} в диапазоне от $0$ до $k$, включительно. 
\item Вторая функция представляет собой процедуру маршрутизации, которая будет передана на каждую из станций после присваивания идентификаторов. Эта функций получает \t{только} следующие входные данные:
\begin{itemize}
  \item $s$, \t{идентификатор} станции, на которой пакет находится в текущий момент,
  \item $t$, \t{идентификатор} станции, которая является пунктом назначения пакета ($t \neq s$),
  \item $c$, список \t{идентификаторов} всех соседей станции $s$.
\end{itemize}
  Функция должна вернуть \t{идентификатор} соседа $s$, куда пакет должен быть передан дальше.
\end{itemize}

В одной из подзадач финальная оценка решения зависит от того, какой максимальный идентификатор будет присвоен какой-либо станции (чем меньше, тем лучше).

\textbf{Детали реализации}

Вы должны реализовать две функции:

\begin{itemize}
\item \t{int[] label(int n, int k, int[] u, int[] v)}
\begin{itemize}
\item $n$: число станций в SIB.
\item $k$: максимальный идентификатор, который может быть использован.
\item $u$ и $v$: два массива размера $n-1$, описывающие соединения. Для каждого $i$ ($0 \leq i \leq n-2$), соединение $i$ находится между станциями $u[i]$ и $v[i]$.
\item Функция должна вернуть массив $L$ длины $n$. Для каждого $i$ ($0 \leq i \leq n-1$) значение $L[i]$ должно содержать идентификатор, присвоенный станции $i$. Все элементы массива $L$ должны быть попарно различны и должны находиться в диапазоне от $0$ до $k$, включительно.
\end{itemize}

\item \t{int find\_next\_station(int s, int t, int[] c)}
\begin{itemize}

\item $s$: идентификатор станции, на которой сейчас находится пакет.
\item $t$: идентификатор станции, которая является пунктом назначения пакета.
\item $c$: массив, содержащий список идентификаторов соседей $s$. Массив $c$ отсортирован по возрастанию.
\item Функция должна вернуть идентификатор соседа $s$, куда пакет должен быть передан дальше.
\end{itemize}
\end{itemize}

Каждый тест содержит один или несколько сценариев (описаний различных вариантов сетей SIB).
Для теста, состоящего из $r$ сценариев, \t{программа}, вызывающая описанные выше функции, будет запущена в точности два раза следующим образом.

Во время первого запуска программы:
\begin{itemize}
\item функция \t{label} будет вызвана $r$ раз,
\item результаты вызовов будут сохранены тестирующей системой, и
\item функция \t{find\_next\_station} не будет вызвана ни разу.
\end{itemize}

Во время второго запуска программы:
\begin{itemize}
\item функция \t{find\_next\_station} может быть вызвана несколько раз,
\item идентификаторы, используемые в функции \t{find\_next\_station}, соответствуют идентификаторам, полученным функцией \t{label} для одного из \t{выбранных произвольным образом} сценариев из первого запуска, и
\item функция \t{label} не будет вызвана ни разу.
\end{itemize}

В частности, никакая информаций, сохраненная в статические или глобальные переменные во время первого запуска, не будет доступна внутри функции  \t{find\_next\_station}.



