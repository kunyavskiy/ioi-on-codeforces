\newcommand{\gt}{\textgreater} 
\newcommand{\lt}{\textless} 


\begin{center}
\renewcommand{\arraystretch}{1.5}
\begin{tabular}{|c|c|c|c|}
\hline
Подзадача & Баллы & $K$ & \parbox{10cm}{\centering \vspace{2mm}Дополнительные ограничения на входные данные\\\vspace{2mm}}\\
\hline
1 & 5 & $k = 1000$ & Ни одна из станции не имеет более $2$ соседей. \\
\hline
2 & 8 & $k = 1000$ & Соединение $i$ соответствует станциям $i+1$ и $\left\lfloor \frac{i}{2} \right\rfloor$.\\
\hline
3 & 16 & $k = 1\,000\,000$ & Не более одной станции имеет более $2$ соседей.\\
\hline
4 & 10 & $k = 10^9$ & $n \leq 8$ \\
\hline
5 & 61 & $k = 10^9$ & --- \\
\hline
\end{tabular}
\end{center}


В подзадаче 5 вы можете получить частичный балл. Пусть $m$ обозначает максимальное значение элементов \t{label} среди всех сценариев. Ваш балл за эту подзадач вычисляется следующим образом:

\begin{center}
\renewcommand{\arraystretch}{1.5}
\begin{tabular}{|c|c|}
\hline
Максимальный идентификатор & Балл  \\
\hline
$m\geq10^9$ & $0$ \\
\hline
$2000 \leq m \lt 10^9$ & $50 \cdot \log_{5\cdot10^5}(\frac{10^9}{m})$  \\
\hline
$1000 \lt m \lt 2000$ & $50$ \\
\hline
$m\leq 1000$ & $61$ \\
\hline

\end{tabular}
\end{center}