Проверяющий модуль выводит данные в следующем формате:
\begin{itemize}
\item строка $1$: $m$
\end{itemize}

Далее следуют $r$ блоков данных, соответствующих сценариям из входных данных. Каждый из блоков имеет следующий формат:
\begin{itemize}
\item строка $1 + j$ ($0 \leq j \leq q - 1$): \t{индекс} станции, чей \t{идентификатор} был получен в результате $j$-го вызова функции \t{find\_next\_station} в данном сценарии.
\end{itemize}

Обратите внимание, что один запуск проверяющего модуля вызывает обе функции \t{label} и \t{find\_next\_station}.