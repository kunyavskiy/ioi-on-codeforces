There is a sequence of $N$ jobs to be processed on one machine. The jobs are numbered
from $1$ to $N$, so that the sequence is $1,2, \ldots, N$. The sequence of jobs must be partitioned into one or more batches, where each batch consists of consecutive jobs in the sequence. The processing starts at time $0$. The batches are handled one by one starting from the first batch as follows. If a batch $b$ contains jobs with smaller numbers than batch $c$, then batch $b$ is handled before batch $c$. The jobs in a batch are processed successively on the machine. Immediately after all the jobs in a batch are processed, the machine outputs the results of all the jobs in that batch. The output time of a job $j$ is the time when the batch containing $j$ finishes.

A setup time $S$ is needed to set up the machine for each batch. For each job $i$, we know
its cost factor $F_i$ and the time $T_i$ required to process it. If a batch contains the jobs $x, x+1, \ldots , x+k$, and starts at time $t$, then the output time of every job in that batch is $t + S + (T_x + T_{x+1} + \ldots + T_{x+k})$. Note that the machine outputs the results of all jobs in a batch at the same time. If the output time of job $i$ is $O_i$, its cost is $O_i \times F_i$. 

For example, assume that there are $5$ jobs, the setup time $S = 1$, $(T_1, T_2, T_3, T_4, T_5) = (1, 3, 4, 2, 1)$, and $(F_1, F_2, F_3, F_4, F_5) = (3, 2, 3, 3, 4)$. If the jobs are partitioned into three batches ${1, 2}$, ${3}$, ${4, 5}$, then the output times $(O_1, O_2, O_3, O_4, O_5) = (5, 5, 10, 14, 14)$ and the costs of the jobs are $(15, 10, 30, 42, 56)$, respectively. The total cost for a partitioning is the sum of the costs of all jobs. The total cost for the example partitioning above is $153$.

You are to write a program which, given the batch setup time and a sequence of jobs
with their processing times and cost factors, computes the minimum possible total cost.