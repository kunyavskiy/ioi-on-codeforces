There is a class of $N$ students, numbered $0$ through $N - 1$. Every day the teacher of the class has some projects for the students. Each project has to be completed by a team of students within the same day. The projects may have various difficulty. For each project, the teacher knows the exact size of a team that should work on it.

Different students may prefer different team sizes. More precisely, student $i$ can only be assigned to a team of size between $A[i]$ and $B[i]$ inclusive. On each day, a student may be assigned to at most one team. Some students might not be assigned to any teams. Each team will work on a single project.

The teacher has already chosen the projects for each of the next $Q$ days. For each of these days, determine whether it is possible to assign students to teams so that there is one team working on each project.

You are given the description of all students: $N$, $A$, and $B$, as well as a sequence of $Q$ questions~--- one about each day. Each question consists of the number $M$ of projects on that day and a sequence $K$
of length $M$ containing the required team sizes. For each question, your program must return whether it is possible to form all the teams.
You need to implement the functions \t{init} and \t{can}:
\begin{itemize}
\item \t{void init(int N, int A[], int B[])}~--- The grader will call this function first and exactly once.
\begin{itemize}
\item $N$: the number of students.
\item $A$: an array of length $N$: $A[i]$ is the minimum team size for student $i$.
\item $B$: an array of length $N$: $B[i]$ is the maximum team size for student $i$
\item The function has no return value.
\item You may assume that $1 \le A[i] \le B[i] \le N$ for each $i = 0, \ldots, N - 1$
\end{itemize}
\item \t{int can(int M, int K[])}~--- After calling \t{init} once, the grader will call this function $Q$ times in a row, once for each day.
\begin{itemize}
\item $M$: the number of projects for this day.
\item $K$: an array of length $M$ containing the required team size for each of these projects.
\item The function should return $1$ if it is possible to form all the required teams and $0$ otherwise.
\item You may assume that $1 \le M \le N$, and that for each $i = 0, \ldots, M - 1$ we have $1 \le K[i] \le N$. Note that the sum of all $K[i]$ may exceed $N$.
\end{itemize}
\end{itemize}