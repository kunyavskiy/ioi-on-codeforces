Полные ограничения:
\begin{itemize}
\item $1 \leq N \leq 100,000$
\item $0 \leq M \leq N ­- 1$
\item $0 \leq A[i], B[i] \leq N ­- 1$
\item $1 \leq T[i] \leq 10,000$
\item $1 \leq L \leq 10,000$
\end{itemize}

\begin{center}
\renewcommand{\arraystretch}{1.5}
\begin{tabular}{|c|c|c|}
\hline
Подзадача & Баллы & Дополнительные ограничения на входные данные\\
\hline
1 & 14 & \parbox{13cm}{\centering \vspace{2mm}$M = N - 2$, и из каждого озера выходит ровно одна или две изначально заданные тропинки. Другими словами, есть два набора связанных между собой озёр, в каждом из которых изначально заданные тропинки соединены в неветвящийся путь.\\\vspace{2mm}} \\
\hline
2 & 10 & $M = N - 2$ и $N \leq 100$ \\
\hline
3 & 23 & $M = N ­- 2$ \\
\hline
4 & 18 & Из каждого озера выходит не более однойизначально заданной тропинки. \\
\hline
5 & 12 & $N \leq 3,000$ \\
\hline
6 & 23 & (Нет) \\
\hline
\end{tabular}
\end{center}
