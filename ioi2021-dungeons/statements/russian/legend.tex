Роберт разрабатывает новую компьютерную игру. В игре участвует один герой, $n$ врагов и $n+1$ подземелье. Враги пронумерованы от $0$ до $n-1$, подземелья пронумерованы от $0$ до $n$. Враг $i$ ($0 \leq i \leq n-1$) находится в подземелье $i$ и имеет силу $s[i]$. В подземелье $n$ врагов нет.

Герой начинает с подземелья $x$, с начальной силой $z$.
Каждый раз, когда герой попадает в какое-то подземелье $i$ ($0 \leq i \leq n-1$), он сражается с врагом $i$, и происходит одно из следующих событий:
\begin{itemize}
\item Если сила героя больше или равна силе врага $s[i]$, герой побеждает. Это приводит к \textbf{увеличению} силы героя на $s[i]$ ($s[i] \geq 1$). В этом случае герой переходит в подземелье $w[i]$ ($w[i] > i$).

\item В противном случае герой проигрывает. Это приводит к  \textbf{увеличению} силы героя на $p[i]$ ($p[i] \geq 1$). В этом случае герой переходит в подземелье $l[i]$.
\end{itemize}


Заметим, что $p[i]$ может быть меньше, равно или больше, чем $s[i]$. Также, 
$l[i]$ может быть меньше, равно или больше, чем $i$.
Независимо от исхода сражения, враг остаётся в подземелье $i$, и сохраняет силу $s[i]$.

Игра заканчивается, когда герой попадает в подземелье $n$.
Можно показать, что игра заканчивается через конечное число сражений, независимо от стартового подземелья и силы героя.

Роберт просит Вас протестировать его игру, проведя $q$ симуляций.
Для каждой симуляции Роберт определил стартовое подземелье $x$ и стартовую силу $z$.
Ваша задача выяснить силу героя в конце игры для каждой симуляции.


\textbf{Детали реализации}

Вы должны реализовать следующие функции:

\begin{itemize}
\item \texttt{void init(int n, int[] s, int[] p, int[] w, int[] l)}
\begin{itemize}
\item $n$: количество врагов.
\item $s$, $p$, $w$, $l$: массивы длины $n$. Для $0 \leq i \leq n-1$:
  \begin{itemize}
  \item $s[i]$ это сила врага $i$, а также сила, получаемая героем после победы над врагом $i$.
  \item $p[i]$ это сила, получаемая героем после поражения от врага $i$.
  \item $w[i]$ это подземелье, куда переходит герой после победы над врагом $i$.
  \item $l[i]$ это подземелье, куда переходит герой после поражения от врага $i$.
\end{itemize}
\item Эта функция вызывается ровно один раз, до вызовов \texttt{simulate} (смотри ниже).
\end{itemize}
\item \texttt{int64 simulate(int x, int z)}
\begin{itemize}
\item $x$: стартовое подземелье для героя.
\item $z$: стартовая сила героя.
\item Эта функция должна вернуть силу героя в конце игры, если герой начинает игру с подземелья $x$, имея силу $z$.
\item Эта функция будет вызвана ровно $q$ раз.
\end{itemize}
\end{itemize}