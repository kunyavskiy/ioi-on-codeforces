Это задача с открытыми тестами. Это означает, что вам не требуется посылать решение, необходимо сдать только ответы на известный заранее набор тестов. Вы можете скачать тесты в секции материалов задачи. Это архив, содержащий файлы 01, 02, 03, \dots. Необходимо сдать один zip-архив содержащий ответы 01.out, 02.out, 03.out, \dots в корне архива. Архив может не содержать ответы на некоторые тесты, в этом случае вы получите вердикт ``\t{Неправильный ответ}'' на этих тестах.

Рассмотрим вычислительную машину, выполняющую две операции (назовем её
МДО). Она имеет девять регистров, пронумерованых числами от $1$ до $9$. Каждый
регистр хранит неотрицательное целое число в диапазоне от $0$ до $1000$ 
включительно. МДО способна выполнять две операции: 

\begin{tabular}{ll}
S i j & Записать значение регистра i, увеличенное на 1, в регистр j
(i может быть равным j). \\
P i & Напечатать значение регистра i. 
\end{tabular}

Программа для МДО состоит из начальных значений каждого регистра и
последовательности операций. Для заданного целого числа $N$ ($0 \le N \le 255$) 
необходимо написать для МДО программу, которая печатает убывающую
последовательность целых чисел $N, N-1, N-2, \ldots, 0$. Максимальное количество
подряд идущих S-операций должно быть как можно меньше. 
Пример программы для МДО и её исполнения при N=2: 

\begin{tabular}{|c|c|c|}
\hline
Операция & Новые значения регистров & Напечатанное значение \\
& 1 2 3 4 5 6 7 8 9 &  \\ \hline
Начальные значения & 0 2 0 0 0 0 0 0 0 & \\ \hline
P 2 & 0 2 0 0 0 0 0 0 0 & 2 \\ \hline
S 1 3 & 0 2 1 0 0 0 0 0 0 & \\ \hline
P 3 & 0 2 1 0 0 0 0 0 0 & 1 \\ \hline
P 1 & 0 2 1 0 0 0 0 0 0 & 0 \\ \hline
\end{tabular}


Варианты входных данных, пронумерованные от 1 до 16, расположены на сервере
соревнования. 