Тетя Хонг организует соревнование с $x$ участниками, и хочет дать каждому участнику по \t{сумке с печеньками}. Есть $k$ разных типов печенек, пронумерованных от $0$ до $k-1$. Каждая из печенек типа $i$ ($0 \leq i \leq k-1$) имеет \t{вкусность}, которая равна $2^i$. У тети Хонг в кладовке есть $a[i]$ (возможно, 0) печенек типа $i$.

Каждая сумка тети Хонг должна содержать ноль или более печенек каждого типа. Суммарное количество печенек типа $i$ во всех сумках не должно превосходить $a[i]$. Сумма вкусностей всех печенек в сумке называется \t{суммарной вкусностью} сумки.

Помогите тете Хонг узнать, как много существует различных значений $y$ таких, что существует способ упаковать $x$ сумок с печеньками, у каждой из которых суммарная вкусность равна $y$. 

\textbf{Детали реализации}

Вам необходимо реализовать следующую функцию:

\begin{itemize}
\item \t{int64 count\_tastiness(int64 x, int64[] a)}
\begin{itemize}
\item $x$: количество сумок с печеньками, которые необходимо упаковать.
\item $a$: массив длины $k$. Для каждого $0 \leq i \leq k-1$ значение $a[i]$ обозначает количество печенек типа $i$ в кладовке.
\item Функция должна вернуть количество различных значений $y$ таких, что тетя может упаковать $x$ сумок с печеньками, у каждой из которых суммарная вкусность равна $y$.
\item Функция будет вызвана $q$ раз (разрешенные значения $q$ описаны в секциях Ограничения и Подзадачи). Каждый из этих вызовов должен рассматриваться независимо. 
\end{itemize}
\end{itemize}

