% Origin: IOI 2005, Poland (http://www.ioi2005.pl/), Day 1, mou
% Translation: Yury Petrov

В парке развлечений <<Ай-ой-ай>> открылся новейший аттракцион: польские горки.
Трек состоит из $n$ рельс, присоединённых одна к концу другой. Начало первой
рельсы находится на высоте $0$. Оператор Петя может конфигурировать
аттракцион, изменяя по своему желанию подъём нескольких последовательных
рельс. При этом подъём всех остальных рельс не изменяется. При каждом
изменении конфигурации рельс положение следующих за изменяемыми подбирается
таким образом, чтобы весь трек оставался связным.

Каждый запуск вагонетки осуществляется с энергией, достаточной для достижения
высоты $h$. Это значит, что вагонетка будет двигаться до тех пор, пока высота
не превысит $h$, либо пока не закончится трек.

По записям о всех изменениях конфигурации рельс и временах запусков вагонетки
для каждого запуска определите, сколько рельс вагонетка проедет до остановки.

Трек можно представить как последовательность $n$ подъёмов $d_i$, по одному
на рельс. Изначально все рельсы горизонтальны, то есть $d_i = 0$ для всех $i$.

\includegraphics{mountains.1}

Каждое изменение конфигурации определяется числами $a$, $b$ и $D$: все рельсы
с $a$-й по $b$-ю включительно после этого действия имеют подъём $D$.

\includegraphics{mountains.2}

\includegraphics{mountains.3}

Каждый запуск вагонетки определяется единственным целым числом $h$ "---
максимальной высотой, на которую способна подняться вагонетка.

