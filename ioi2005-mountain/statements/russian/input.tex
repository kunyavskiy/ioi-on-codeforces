В первой строке записано целое число $n$ ($1 \le n \le 10^9$) "--- число
рельс. Следующие строки содержат запросы трёх видов:
\begin{itemize}
  \item \texttt{I} $a$ $b$ $D$ "--- изменение конфигурации. Рельсы с $a$-й по $b$-ю
    включительно после выполнения запроса имеют подъём, равный $D$. ($1 \le a \le b \le n$).
  \item \texttt{Q} $h$ "--- запуск вагонетки. Требуется найти число рельс, которое
    проедет вагонетка, которая способна подняться на высоту $h$. ($0 \le h \le 10^9$).
  \item \texttt{E} "--- конец ввода. Этот запрос встретится ровно один раз в конце
    файла.
\end{itemize}

В любой момент высота любой точки трека лежит от $0$ до $10^9$. Во вводе не
более $100\,000$ строк.

