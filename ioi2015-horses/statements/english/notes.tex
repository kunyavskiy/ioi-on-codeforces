Suppose that there are $N = 3$ years, with the following information:
\begin{center}
\begin{tabular}{|c|c|c|c|}
\hline
  & 0 & 1 & 2\\
\hline
X & 2 & 1 & 3\\
\hline
Y & 3 & 4 & 1\\
\hline
\end{tabular}
\end{center}

For these initial values, Mansur can earn the most if he sells both his horses at the end of year 1. The entire process will look as follows:
\begin{itemize}
\item Initially, Mansur has 1 horse.
\item After year 0 he will have $1 \cdot X[0] = 2$ horses.
\item After year 1 he will have $2 \cdot X[1] = 2$ horses.
\item He can now sell those two horses. The total profit will be $2 \cdot Y[1] = 8$.
\end{itemize}

Then, suppose that there is $M = 1$ update: changing $Y[1]$ to 2.

After the update we will have:
\begin{center}
\begin{tabular}{|c|c|c|c|}
\hline
  & 0 & 1 & 2\\
\hline
X & 2 & 1 & 3\\
\hline
Y & 3 & 2 & 1\\
\hline
\end{tabular}
\end{center}
In this case, one of the optimal solutions is to sell one horse after year 0 and then three horses after year 2. The entire process will look as follows:
\begin{itemize}
\item Initially, Mansur has 1 horse.
\item After year 0 he will have $1 \cdot X[0] = 2$ horses.
\item He can now sell one of those horses for $Y[0] = 3$, and have one horse left.
\item After year 1 he will have $1 \cdot X[1] = 1$ horse.
\item After year 2 he will have $1 \cdot X[2] = 3$ horses.
\item He can now sell those three horses for $3 \cdot Y[2] = 3$. The total amount of money is $3 + 3 = 6$.
\end{itemize}
