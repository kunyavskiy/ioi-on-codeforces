Рамзес Второй только что вернулся с победоносной битвы. Чтобы увековечить свою
победу, он решил построить величественный сад. Сад будет содержать длинную
линию из растений, которая будет тянуться вдоль всего пути от его дворца в
Луксоре до Карнакского храма. Сад будет состоять только из лотосов и папирусов,
поскольку они символизируют Верхний и Нижний Египет соответственно.

Сад должен содержать ровно $N$ растений.
Кроме того, он должен быть сбалансирован: на любом непрерывном отрезке сада
количества лотосов и папирусов не должны отличаться более, чем на 2.

Сад может быть представлен в виде строки букв <<\texttt{L}>> (лотос)
и <<\texttt{P}>> (папирус).
Например, для $N = 5$ возможны 14 сбалансированных садов.
В алфавитном порядке это сады:
\texttt{LLPLP}, \texttt{LLPPL}, \texttt{LPLLP}, \texttt{LPLPL}, \texttt{LPLPP},
\texttt{LPPLL}, \texttt{LPPLP}, \texttt{PLLPL}, \texttt{PLLPP}, \texttt{PLPLL},
\texttt{PLPLP}, \texttt{PLPPL}, \texttt{PPLLP} и \texttt{PPLPL}.

Возможные сбалансированные сады данной длины могут быть упорядочены в алфавитном 
порядке и пронумерованы, начиная с 1. Например, для $N = 5$ сад с номером $12$ "--- это 
сад \texttt{PLPPL}.

Напишите программу, которая по заданному количеству растений $N$ и строке,
которая представляет сбалансированный сад, вычисляет номер,
присвоенный этому саду.
Так как он может быть достаточно большим,
его следует вывести по модулю заданного целого числа $M$.
