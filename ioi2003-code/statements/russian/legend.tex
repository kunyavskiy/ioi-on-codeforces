Компания Racine Business Networks (RBN) подала в суд на компанию Heuristic 
Algorithm Languages (HAL), утверждая что HAL использовала исходный код из
RBN UNIX(TM) и внесла его в открытый код операционной системы HALinux. 

Как RBN, так и HAL используют язык программирования, в котором каждый
оператор располагается на отдельной строке и имеет вид: 

$$ STOREA = STOREB + STOREC $$

(\texttt{STOREA}, \texttt{STOREB} и \texttt{STOREC}~--- имена переменных). Каждый оператор
записывается следующим образом: имя первой переменной начинается с первой
позиции строки, затем~--- пробел, знак равенства, пробел, имя второй переменной, 
пробел, знак сложения, пробел и имя третьей переменной. Одно и то же имя
переменной может встречаться в строке более одного раза. Имена переменных 
имеют длину от 1 до 8 символов и состоят из заглавных латинских ASCII букв
(<<A>> \ldots <<Z>>). 

Утверждается, что HAL скопировала последовательность строк программы прямо
из исходного кода RBN с минимальными изменениями. 
\begin{itemize}
\item Чтобы скрыть свое преступление, HAL изменила имена некоторых
переменных. Точнее, HAL взяла последовательность строк из программы
RBN и для каждой переменной в ней изменила ее имя. При этом новое имя
переменной может совпасть со старым. После переименования никакие две
различные переменные не могут называться одинаково. 
\item HAL могла изменить в некоторых строках порядок имен переменных в
правой части оператора присваивания. Например, оператор вида
$STOREA = STOREB + STOREC$ 
мог быть изменен так: 
$STOREA = STOREC + STOREB$
\item Утверждается также, что HAL не изменила порядок, в котором строки кода
следуют в исходном тексте программы. 
\end{itemize}

Необходимо по заданным исходным кодам программ RBN и HAL найти самую
длинную последовательность подряд идущих строк из программы HAL, которую
можно получить из последовательности подряд идущих строк программы RBN с
помощью указанных выше преобразований. Заметьте, что найденная
последовательность в коде HAL и соответствующая ей последовательность в коде
RBN не обязательно начинаются в строках с одинаковыми номерами. 