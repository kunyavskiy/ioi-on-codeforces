В сельской местности находится длинная прямая дорога, которая известна как Рисовый Путь. Вдоль этой дороги расположены $R$ рисовых полей. Каждое поле имеет целочисленную координату от $1$ до $L$ включительно. Рисовые поля задаются в порядке неубывания их координат. Формально, обозначим для $0 \le i < R$ координату рисового поля с номером $i$ как $X[i]$. Гарантируется, что $1 \le X[0] \le \ldots \le X[R-1] \le L$.

Отметим, что несколько рисовых полей могут иметь одинаковую координату.

Планируется построить одно рисохранилище, в которое требуется завезти с полей как можно больше риса. Как и рисовые поля, рисохранилище должно иметь целочисленную координату от $1$ до $L$ включительно. Рисохранилище разрешается строить в любой целочисленной координате, в том числе и в координате, в которой уже есть одно или более рисовых полей.

С каждого рисового поля в течение каждого сезона снимают урожай, который помещается ровно в $1$ грузовике. Чтобы доставить урожай в рисохранилище, необходимо нанять водителя грузовика. Стоимость работы водителя по перевозке груза на единицу расстояния составляет $1$ бат. Другими словами, стоимость транспортировки риса от заданного поля до рисохранилища равна модулю разности их координат.

К сожалению, бюджет рассматриваемого сезона ограничен: нельзя потратить более $B$ бат на транспортировку. Необходимо построить рисохранилище в таком месте, чтобы можно было завезти в него как можно больше риса.

Напишите процедуру \t{besthub(R,L,X,B)}, которой передаются следующие параметры:

\begin{itemize}
\item $R$~--- количество рисовых полей. Поля пронумерованы от $0$ до $R-1$.
\item $L$~--- максимальная координата.
\item $X$~--- одномерный массив целых чисел, отсортированный в порядке неубывания. Для каждого $i$ $0 \le i < R$ рисовое поле с номером $i$ имеет координату $X[i]$.
\item $B$~--- бюджет.
\end{itemize}

Ваша процедура должна находить оптимальное расположение рисохранилища и возвращать максимальное количество грузовиков риса, которые могут быть перевезены в рисохранилище, не превышая заданный бюджет.

Необходимо обратить внимание, что цена транспортировки риса может быть очень большой. Бюджет задаётся 64-битным целым числом, и поэтому рекомендуется использовать 64- битные целые числа для вычислений. В языках C/C++ используйте тип \t{long long}; в языке Паскаль используйте тип \t{Int64}.
