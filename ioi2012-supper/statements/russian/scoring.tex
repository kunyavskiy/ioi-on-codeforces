\begin{tabular}{|c|c|lll|}
\hline
\bf{Подзадача}&\bf{Баллы}&\multicolumn{3}{c|}{\bf{Ограничения}}\\\hline
1 & 8 & $N \le 5\,000$ & $M = 65\,000\,бит$&\\\hline
2 & 9 & $N \le 100\,000$ & $M = 2\,000\,000\,бит$&\\\hline
3 & 9 & $N \le 100\,000$ & $M = 1\,500\,бит$ & $K \le 25\,000$\\\hline
4 & 35 & $N \le 5\,000$ & $M = 10\,000\,бит$&\\\hline
5 & до 39 & $N \le 100\,000$ & $M = 1\,800\,000\,бит$ & $K \le 25\,000$\\\hline
\end{tabular}

Количество баллов за эту подзадачу зависит от длины R последовательности-подсказки, которую составила ваша программа. В частности, если $R_{max}$ - это самое большое (по всем тестам) значение длины последовательности-подсказки созданной процедурой \t{ComputeAdvice}, количество балов будет равно:
\begin{itemize}
\item $39$ баллов, если $R_{max} \le 200\,000$;
\item $\frac{39*(1\,800\,000 - R_{max})}{1\,600\,000}$ баллов, если $200\,000 < R_{max} < 1\,800\,000$;
\item $0$ баллов, если $R_{max} \geq 1\,800\,000$.
\end{itemize}