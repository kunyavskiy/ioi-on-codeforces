Ваша программа должна читать из стандартного потока ввода следующие данные:
\begin{itemize}
\item Первая строка содержит в указанном порядке целые числа $N$, $R$ и $Q$ ($1 \le N \le 200\,000$, $1 \le R \le 25\,000$, $1 \le Q \le 200\,000$), разделённые одиночными
пробелами.
\item Последующие $N$ строк описывают $N$ работников агентства в порядке от старших к младшим. $k$-я из этих $N$ строк описывает работника с номером $k$. Первая из этих строк (то есть, та, которая описывает председателя) содержит одно целое число $H_1$~--- номер региона, из которого приехал председатель. Каждая из последующих $N-1$ строк содержит два целых числа, разделённых одним пробелом: целое число $S_k$ ($1\le S_k < k$)~--- номер непосредственного начальника для работника с номером $k$ и целое число $H_k$ ($1 \le H_k \le R$)~--- номер региона, из которого приехал работник с номером $k$.
\end{itemize}

После прочтения входных данных ваша программа должна начать в интерактивном режиме читать запросы из стандартного потока ввода и выводить результаты обработки запросов в стандартный поток вывода. На каждый из $Q$ запросов необходимо отвечать отдельно, то есть, ваша программа должна вывести результат обработки принятого запроса до получения следующего запроса.

Каждый запрос представлен в одной строке стандартного потока ввода и состоит из двух целых чисел~--- номеров регионов $r_1$ и $r_2$ ($1 \le r_1, r_2 \le R$). Эти числа разделены одним пробелом.

Ответом на каждый запрос является одна строка, выведенная в стандартный поток вывода. Эта строка должна содержать одно целое число~--- количество пар работников АРРООН ($e_1$, $e_2$) таких, что работник с номером $e_1$ приехал из региона с номером $r_1$, работник с номером $e_2$ приехал из региона с номером $r_2$ и работник с номером $e_1$ является менеджером работника с номером $e_2$.

Тесты будут такими, что правильный ответ на любой запрос, задаваемый в стандартном
потоке ввода, всегда будет меньше $1\,000\,000\,000$.