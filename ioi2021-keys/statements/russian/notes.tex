\textbf{Пример 1}

Рассмотрим следующий вызов функции
\begin{verbatim}
find_reachable([0, 1, 1, 2],
               [0, 0, 1, 1, 3], 
               [1, 2, 2, 3, 1], 
               [0, 0, 1, 0, 2])
\end{verbatim}

Если игрок начинает в комнате $0$, то игрок может выполнить следующую последовательность действий:

\begin{center}
\renewcommand{\arraystretch}{1.5}
\begin{tabular}{|c|c|}
\hline
Текущая комната & Действие \\
\hline
$0$ & Собрать ключ типа $0$ \\
\hline
$0$ & Перейти по переходу $0$ в комнату $1$ \\
\hline
$1$ & Собрать ключ типа $1$ \\
\hline
$1$ & Перейти по переходу $2$ в комнату $2$ \\
\hline
$2$ & Перейти по переходу $2$ в комнату $1$ \\
\hline
$1$ & Перейти по переходу $3$ в комнату $3$ \\
\hline
\end{tabular}
\end{center}

Таким образом, комната $3$ достижима из комнаты $0$.
Аналогичным образом мы можем построить последовательности действий для каждой из комнат, таким образом, все они являются достижимыми из комнаты $0$, что означает, что  $p[0]=4$.
В таблице ниже приведены достижимые комнаты для каждого варианта стартовой комнаты:

\begin{center}
\renewcommand{\arraystretch}{1.5}
\begin{tabular}{|c|c|c|}
\hline
Стартовая комната $i$ & Достижимые комнаты & $p[i]$ \\
\hline
$0$ & $[0,1,2,3]$ & $4$\\
\hline
$1$ & $[1,2]$ & $2$\\
\hline
$2$ & $[1,2]$ & $2$\\
\hline
$3$ & $[1,2,3]$ & $3$\\
\hline
\end{tabular}
\end{center}
Наименьшее значение $p[i]$ среди всех комнат равно $2$, и оно достигается для $i=1$ и $i=2$. Таким образом, функция должна вернуть $[0,1,1,0]$.



\textbf{Пример 2}
\begin{verbatim}
find_reachable([0, 1, 1, 2, 2, 1, 2],
               [0, 0, 1, 1, 2, 3, 3, 4, 4, 5],
               [1, 2, 2, 3, 3, 4, 5, 5, 6, 6],
               [0, 0, 1, 0, 0, 1, 2, 0, 2, 1])
\end{verbatim}
В таблице ниже приведены достижимые комнаты для каждого варианта стартовой комнаты:

\begin{center}
\renewcommand{\arraystretch}{1.5}
\begin{tabular}{|c|c|c|}
\hline
Стартовая комната $i$ & Достижимые комнаты & $p[i]$ \\
\hline
$0$ & $[0,1,2,3,4,5,6]$ & $7$\\
\hline
$1$ & $[1,2]$ & $2$\\
\hline
$2$ & $[1,2]$ & $2$\\
\hline
$3$ & $[3,4,5,6]$ & $4$\\
\hline
$4$ & $[4,6]$ & $2$\\
\hline
$5$ & $[3,4,5,6]$ & $4$\\
\hline
$6$ & $[4,6]$ & $2$\\
\hline
\end{tabular}
\end{center}

Наименьшее значение $p[i]$ среди всех комнат равно $2$, и это значение достигается для $i \in \\{1,2,4,6\\}$. Таким образом, функция должна вернуть $[0,1,1,0,1,0,1]$.

\textbf{Пример 3}

\texttt{find\_reachable([0, 0, 0], [0], [1], [0])}

В таблице ниже приведены достижимые комнаты для каждого варианта стартовой комнаты:

\begin{center}
\renewcommand{\arraystretch}{1.5}
\begin{tabular}{|c|c|c|}
\hline
Стартовая комната $i$ & Достижимые комнаты & $p[i]$ \\
\hline
$0$ & $[0,1]$ & $2$\\
\hline
$1$ & $[0,1]$ & $2$\\
\hline
$2$ & $[2]$ & $1$\\
\hline
\end{tabular}
\end{center}

Наименьшее значение $p[i]$ среди всех комнат равно $1$, и это значение достигается при $i=2$. Таким образом, функция должна вернуть $[0,0,1]$.


