Петр работает в компании, которая создала прибор для обнаружения молекул. Каждая молекула имеет целый положительный вес. Прибор характеризуется \textit{интервалом обнаружения} $[l, u]$,где $l$ и $u$ целые положительные числа. Прибор может обнаружить множество молекул тогда и только тогда, когда это множество содержит такое подмножество, что суммарный вес молекул в нем принадлежит интервалу обнаружения прибора.

Более формально, рассмотрим $n$ молекул с весами $w_0, \ldots, w_{n - 1}$.Обнаружение считается успешным, если существует множество различных индексов $I = {i_1, \ldots , i_m}$ такое, что $l \le w_{i_1} + \ldots + w_{i_m} \le u$.

В силу особенностей работы прибора разница между $l$ и $u$ гарантированно больше либо равна разнице весов между самой тяжелой и самой легкой молекулами. Более формально, $u - l \ge w_{max} - w_{min}$, где $w_{max}=\max(w_0, \ldots, w_{n - 1})$ и $w_{min}=\min(w_0, \ldots, w_{n - 1})$.

Требуется написать программу, которая либо находит любое подмножество молекул с суммарным весом, принадлежащим интервалу обнаружения прибора, либо определяет, что такого подмножества не существует.

\textbf{Детали реализации}

Вам следует реализовать одну функцию (метод):
\begin{itemize}
\item \texttt{int[] solve(int l, int u, int[] w)}

\begin{itemize}
\item \texttt{l} и \texttt{u}: границы интервала обнаружения,
\item \texttt{w}:весамолекул.
\item Если требуемое подмножество существует, то функция должна вернуть массив индексов молекул, которые формируют любое такое подмножество. Если существует несколько правильных ответов, верните любой из них.
\item Если требуемого подмножества не существует, то функция должна вернуть пустой массив.
\end{itemize}
\end{itemize}

Ваша программа может записывать индексы в возвращаемый массив в любом порядке.

Пожалуйста, используйте предоставленные шаблоны файлов для уточнения реализации на выбранном вами языке программирования.