\textbf{Пример 1}

\texttt{solve(15, 17, [6, 8, 8, 7])}

В этом примере есть четыре молекулы с весами 6, 8, 8 и 7. Прибор может обнаружить подмножества молекул с суммарным весом от 15 до 17 включительно. Обратите внимание, что $17-15\geq 8-6$. Суммарный вес
молекул 1 и 3 равен $w_1 + w_3 = 8 + 7 = 15$, таким образом функция может вернуть \texttt{[1, 3]}. Другие возможные правильные ответы: \texttt{[1, 2]} ($w_1 + w_2 = 8 + 8 = 16$) и \texttt{[2, 3]} ($w_2 + w_3 = 8 + 7 = 15$).

\textbf{Пример 2}

\texttt{solve(14, 15, [5, 5, 6, 6])}

В этом примере есть четыре молекулы с весами 5, 5, 6 и 6. Требуется найти подмножество с суммарным весом от 14 до 15 включительно. Опять же, обратите внимание, что $15 - 14 \geq 6 - 5$. Для данного примера не существует
подмножества молекул с суммарным весом от $14$ до $15$, соответственно функция должна вернуть пустой массив.

\textbf{Пример 3}

\texttt{solve(10, 20, [15, 17, 16, 18])}

В этом примере есть четыре молекулы с весами 15, 17, 16 и 18. Требуется найти подмножество с суммарным весом от 10 до 20 включительно. Вновь, обратите внимание, что $20 - 10 \geq 18 - 15$. Любое подмножество, состоящее из одного элемента, имеет вес от $10$ до $20$, соответственно возможные правильные ответы это \texttt{[0]}, \texttt{[1]}, \texttt{[2]} и \texttt{[3]}.