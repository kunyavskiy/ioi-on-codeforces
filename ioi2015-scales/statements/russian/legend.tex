У Амины есть $6$ монет, пронумерованных от $1$ до $6$. Она знает, что все монеты имеют разный
вес. Амина хочет упорядочить монеты по весу. Для этого у нее есть специальные весы.

У классических весов есть две чаши. Для того, чтобы воспользоваться ими, необходимо
положить по одной монете на каждую чашу весов, и весы определят, какая из монет тяжелее.

Новые весы Амины устроены сложнее. У них есть четыре чаши, обозначенные $A$, $B$, $C$ и $D$.
Эти весы работают в четырех режимах, в каждом из которых они отвечают на различные
вопросы о положенных на них монетах. В первых трех режимах Амина должна положить ровно
по одной монете на каждую из чаш $A$, $B$ и $C$. При использовании четвертого режима она
должна, кроме чаш $A$, $B$ и $C$, дополнительно положить ровно одну монету на чашу $D$.

Четыре режима отвечают на следующие четыре вопроса соответственно:
\begin{enumerate}
\item Какая из монет на чашах $A$, $B$, $C$ самая тяжелая?
\item Какая из монет на чашах $A$, $B$, $C$ самая легкая?
\item Какая из монет на чашах $A$, $B$, $C$ средняя по весу?
\item Среди монет на чашах $A$, $B$, $C$ рассматриваются только монеты, которые тяжелее
монеты на чаше $D$. Если такие монеты есть, то весы сообщают, какая из них самая
легкая. Иначе, если таких монет нет, то весы сообщают, какая из монет на чашах $A$, $B$,
$C$ самая легкая.
\end{enumerate}
Напишите программу, которая расположит шесть монет Амины по весу.

Программа может задавать вопросы весам Амины. Программа должна решить задачу для
нескольких наборов входных данных, каждый из которых соответствует новому набору из
шести монет.

Необходимо реализовать две функции $init$ и $orderCoins$. Во время каждого запуска
программы сначала ровно один раз будет вызвана функция $init$, которой сообщается
количество наборов монет в этом тесте, а также может выполняться необходимая
инициализация переменных. После этого будет вызвана функция $orderCoins()$ один раз для
каждого набора монет.

\begin{itemize}
\item $void\ init(int\ T)$
\begin{itemize}
\item $T$~--- количество наборов монет, для которых необходимо решить задачу в этом
тесте. $T$~--- целое число в промежутке от $1$ до $18$.
\item Эта фунция не возвращает никакого значения.
\end{itemize}
\item $void\ orderCoins()$
\begin{itemize}
\item Эта функция будет вызвана один раз для каждого набора монет.
\item Эта функция должна определить правильный порядок монет Амины, используя
функции $getHeaviest()$, $getLightest()$, $getMedian()$, и/или
$getNextLightest()$.
\item Когда удалось восстановить правильный порядок, необходимо вызвать функцию
$answer()$.
\item После вызова функции $answer()$, функция $orderCoins()$ должна завершиться.
\item Эта функция не возвращает никакого значения.
\end{itemize}
\end{itemize}



В своей программе вы можете использовать следующие функции:

\begin{itemize}
\item $answer(W)$~--- эту функцию необходимо вызвать, чтобы сообщить найденный ответ.
\begin{itemize}
\item $W$~--- массив из $6$ элементов содержащий правильный порядок монет. Значения от
$W[0]$ до $W[5]$ должны быть номерами монет, то есть, числами от $1$ до $6$ в порядке
от самой легкой до самой тяжелой.
\item Эту функцию можно вызывать только один раз в каждом запуске функции
$orderCoins()$.
\item Эта функция не возвращает никакого значения.
\end{itemize}
\item $getHeaviest(A, B, C)$, $getLightest(A, B, C)$, $getMedian(A, B, C)$ --- эти
функции соответствуют 1, 2 и 3 режимам работы весов Амины.
\begin{itemize}
\item $A$, $B$, $C$~--- номера монет в чашах $A$, $B$ и $C$ соответственно. $A$, $B$ и $C$ должны быть
тремя различными целыми числами от $1$ до $6$ включительно.
\item Каждая из функций возвращает одно из чисел A, B, C, соотвествующее подходящей
монете. Например, $getHeaviest(A, B, C)$ возвращает номер самой тяжелой из
трех переданных монет.
\end{itemize}
\item $getNextLightest(A, B, C, D)$~--- эта функция соответствует четвертому режиму
работы весов Амины
\begin {itemize}
\item $A$, $B$, $C$, $D$~--- номера монет, в чашах $A$, $B$, $C$, $D$ соответственно. $A$, $B$, $C$ и $D$ должны
быть четырьмя различным целыми числами от 1 до 6 включительно.
\item Эта функция возвращает одно из чисел A, B или C, выбранное способом, описанным
выше для четвертого режима работы. То есть, это номер самой легкой монеты на
чашах $A$, $B$, $C$, которая тяжелее, чем монета на чаше $D$. Если ни одна из них не
тяжелее монеты на чаше $D$, то возвращается самая легкая из монет на чашах $A$,
$B$, $C$.
\end{itemize}
\end{itemize}