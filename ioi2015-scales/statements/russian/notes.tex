Пусть монеты упорядочены от легких к тяжелым как $3\ 4\ 6\ 2\ 1\ 5$.
\begin{center}
\renewcommand{\arraystretch}{1.5}
\begin{tabular}{|c|c|c|}
\hline
Вызов функции & Значение & Пояснение \\
\hline
$getMedian(4,5,6)$ & $6$ & Монета $6$ является средней по веу среди монет $4$, $5$ и $6$.\\
\hline
$getHeaviest(3,1,2)$ & $1$ & Монета $1$ самая тяжелая среди $1$, $2$ и $3$.\\
\hline
$getNextLightest(2,3,4,5)$ & $3$ & \parbox{10cm}{\centering \vspace{2mm}Монеты $2$, $3$ и $4$ легче, чем $5$, поэтому будет возвращена самая легкая из них (3). \\\vspace{2mm}}\\
\hline
$getNextLightest(1,6,3,4)$ & $6$ & \parbox{10cm}{\centering \vspace{2mm}Монеты $1$ и $6$ обе тяжелее, чем монета $4$. Среди них, будет возвращена более легкая монета $6$.\\\vspace{2mm}}\\
\hline
$getHeaviest(3,5,6)$ & $5$ & Монета $5$ самая тяжелая среди $3$, $5$ и $6$.\\
\hline
$getMedian(1,5,6)$ & $1$ & Монета $1$ средняя по весу, среди $1$, $5$ и $6$.\\
\hline
$getMedian(2,4,6)$ & $6$ & Монета $6$ средняя по весу, среди $2$, $4$ и $6$.\\
\hline
$answer([3,4,6,2,1,5])$ & & Программа нашла корректный ответ для этого набора монет.\\
\hline
\end{tabular}
\end{center}
