Доктор Блэк убит. Детектив Джилл должна определить убийцу, место и орудие убийства. 
Известны шесть подозреваемых, пронумерованных от 1 до 6, десять возможных мест убийства, 
пронумерованных от 1 до 10 и шесть возможных орудий убийства, пронумерованных от 1 до 6. 

\begin{center}
\renewcommand{\arraystretch}{1.5}
\begin{tabular}{|c|c|c|}
\hline
Murderer & Location & Weapon \\
\hline
\parbox{5cm}{\centering \vspace{2mm}
\begin{enumerate}
\item Профессор Плам
\item Мисс Скарлет
\item Полковник Мустард
\item Миссис Уайт
\item Преподобный отец Грин
\item Миссис Пикук
\end{enumerate}---\\
\vspace{2mm}} & 
\parbox{5cm}{\centering \vspace{2mm}
\begin{enumerate}
\item Зал
\item Кухня
\item Оранжерея
\item Столовая
\item Бильярдная
\item Библиотека
\item Гостиная
\item Холл
\item Кабинет
\item Подвал
\end{enumerate}~--- \\
\vspace{2mm}} &
\parbox{5cm}{\centering \vspace{2mm}
\begin{enumerate}
\item Труба
\item Кинжал
\item Подсвечник
\item Револьвер
\item Веревка
\item Монтировка
\end{enumerate}~--- \\
\vspace{2mm}} \\
\hline
\end{tabular}
\end{center}



При расследовании Джилл делает последовательные попытки определения корректной
комбинации из тройки чисел, обозначающих убийцу, место и орудие убийства. Каждая
попытка называется теорией. При каждой попытке Джилл просит своего ассистента
Джека подтвердить или опровергнуть теорию. Когда Джек подтверждает теорию, Джилл
успешно завершает расследование. Если Джек опровергает теорию, то он сообщает
Джилл, что неверно одно из чисел в теории, задающих убийцу, место или орудие
убийства.

Вам необходимо написать процедуру \t{Solve}, которая играет роль Джилл. Система
оценивания будет многократно вызывать процедуру \t{Solve}, каждый раз для нового
расследования. Процедура \t{Solve} должна периодически вызывать процедуру
\t{Theory(M, L, W)}, которая реализована в системе оценивания. \t{M}, \t{L} и \t{W} являются числами,
задающими конкретную комбинацию убийцы, места и орудия убийства. Процедура
\t{Theory(M, L, W)} возвращает значение \t{0}, если теория верна. Если теория ошибочна,
возвращается значение \t{1}, \t{2} или \t{3}. Значение \t{1} означает, что неверно число, задающее
убийцу, значение \t{2} означает, что неверно число, задающее место убийства, значение \t{3}
означает, что неверно число, задающее орудие убийства. Если неверно более одного
числа, Жак выбирает одно произвольное число из неверных, не обязательно
детерминированным образом. Когда процедура \t{Theory(M, L, W)} вернет значение \t{0},
процедура \t{Solve} должна завершиться. 

Для примера представим, что Мисс Скарлет (Подозреваемый №2) совершила убийство в
оранжерее (Место №3), используя револьвер (Орудие №4). Если процедура \t{Solve}
выполнит следующие вызовы процедуры \t{Theory}, то будут возвращены значения,
показанные во втором столбце таблицы ниже. 

\begin{tabular}{|l|l|l|}
\hline
Вызов & Возвращаемый результат & Пояснение \\ \hline
\t{Theory(1, 1, 1)} & 1, 2 или 3 & Все три числа ошибочны \\
\t{Theory(3, 3, 3)} & 1 или 3 & Только место верно \\
\t{Theory(5, 3, 4)} & 1 & Ошибка только в убийце \\
\t{Theory(2, 3, 4)} & 0 & Все верно \\ \hline
\end{tabular}
