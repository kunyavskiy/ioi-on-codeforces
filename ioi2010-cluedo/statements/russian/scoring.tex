\begin{center}
\renewcommand{\arraystretch}{1.5}
\begin{tabular}{|c|c|c|}
\hline
Подзадача & Баллы & \parbox{13cm}{\centering \vspace{2mm}Дополнительные ограничения на входные данные\\\vspace{2mm}}\\
\hline
1 & 50 & \parbox{13cm}{\centering \vspace{2mm}В каждом тесте процедура \t{Solve} вызывается не более $100$ раз. Вызовы могут
соответствовать различным комбинациям убийцы, места и орудия убийства. При каждом
вызове процедура \t{Solve} должна определить верную теорию, используя не более $360$
вызовов процедуры \t{Theory(M, L, W)}. Не забудьте проинициализировать все переменные
при каждом вызове процедуры \t{Solve}.\\\vspace{2mm}}\\
\hline
2 & 50 & \parbox{13cm}{\centering \vspace{2mm} В каждом тесте процедура \t{Solve} вызывается не более $100$ раз. При каждом вызове
процедура \t{Solve} должна определить верную теорию, используя не более $20$ вызовов
процедуры \t{Theory(M, L, W)}. Не забудьте проинициализировать все переменные при
каждом вызове процедуры \t{Solve}.\\\vspace{2mm}} \\
\hline
\end{tabular}
\end{center}