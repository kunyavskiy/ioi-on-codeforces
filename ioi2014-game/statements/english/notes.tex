We explain the game rules with three examples. Each example has $n = 4$ cities and $r = 6$ rounds of question and answer.

In the first example (the following table), Jian-Jia \texttt{loses} because after round 4, Mei-Yu knows for certain that one can travel between any two cities by flights, no matter how Jian-Jia answers questions 5 or 6.

\begin{center}
\renewcommand{\arraystretch}{1.5}
\begin{tabular}{|c|c|c|}
\hline
Round & Question & Answer \\
\hline
1 &  0, 1 & yes\\
\hline
2 &  3, 0 & yes\\
\hline
3 &  1, 2 & no\\
\hline
4 &  0, 2 & yes\\
\hline
--- &  --- & ---\\
\hline
5 &  3, 1 & no\\
\hline
6 &  2, 3 & no\\
\hline
\end{tabular}
\end{center}

In the next example Mei-Yu can prove after round 3 that no matter how Jian-Jia answers questions 4, 5, or 6, one \texttt{cannot} travel between cities 0 and 1 by flights, so Jian-Jia loses again.

\begin{center}
\renewcommand{\arraystretch}{1.5}
\begin{tabular}{|c|c|c|}
\hline
Round & Question & Answer \\
\hline
1 &  0, 3 & no\\
\hline
2 &  2, 0 & no\\
\hline
3 &  0, 1 & no\\
\hline
--- &  --- & ---\\
\hline
4 &  1, 2 & yes\\
\hline
5 &  1, 3 & yes\\
\hline
6 &  2, 3 & yes\\
\hline
\end{tabular}
\end{center}

In the final example Mei-Yu cannot determine whether one can travel between any two cities by flights until allsix questions are answered, so Jian-Jia \texttt{wins} the game. Specifically, because Jian-Jia answered yes to the last question (in the following table), then it is possible to travel between any pair of cities. However, if Jian-Jia had answered no to the last question instead then it would be impossible.

\begin{center}
\renewcommand{\arraystretch}{1.5}
\begin{tabular}{|c|c|c|}
\hline
Round & Question & Answer \\
\hline
1 &  0, 3 & no\\
\hline
2 &  1, 0 & yes\\
\hline
3 &  0, 2 & no\\
\hline
4 &  3, 1 & yes\\
\hline
5 &  1, 2 & no\\
\hline
6 &  2, 3 & yes\\
\hline
\end{tabular}
\end{center}