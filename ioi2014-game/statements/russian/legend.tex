Маленькому мальчику Джан-Джи нравится играть. Когда ему задают вопрос, вместо простого
ответа он предлагает сыграть в игру.

Джан-Джи встретил свою подружку Мей-Ю и рассказал ей о том, как устроена сеть
авиарейсов в Тайване. В частности, в Тайване есть $n$ городов, пронумерованных $0, \ldots, n - 1$. Между некоторыми из них действуют авиарейсы, каждый из них действует в обе
стороны.

Мей-Ю спросила у Джан-Джи, правда ли, что возможно добраться из любого города в любой
другой, используя только авиарейсы. Джан-Джи вместо ответа на вопрос предложил сыграть в
игру.

Мей-Ю может задавать вопросы типа: <<Соединены ли города $x$ и $y$ авиарейсом \texttt{напрямую}?>>. Джан-Джи должен сразу отвечать на каждый из вопросов. Мей-Ю спрашивает про каждую пару городов ровно один раз, то есть, задает всего $r = \frac{n(n - 1)}{2}$ вопросов. Считается, что МейЮ победила в игре, если после получения ответов на первые $i$ вопросов для некоторого $i < r$, она может определить, правда ли, что сеть авиарейсов связна. Это означает, что возможно добраться из любого города в любой другой, используя только авиарейсы. Если ей понадобилось задать все $r$ вопросов, то выиграл Джан-Джи.

Чтобы сделать игру более интересной, друзья договорились, что они забудут о действующей
сети авиарейсов в Тайване, а придумают свою сеть в процессе игры, выбирая ответ на
текущий вопрос в зависимости от предыдущих вопросов Мей-Ю. Необходимо написать
программу, которая поможет Джан-Джи выиграть игру, решая, как ему следует отвечать на
каждый вопрос.

\textbf{Постановка задачи}

Напишите программу, которая поможет Джан-Джи победить в игре. Обратите внимание, что
Мей-Ю и Джан-Джи не знают стратегию друг друга. Мей-Ю может спрашивать про пары
городов в любом порядке, и Джан-Джи должен отвечать на каждый вопрос сразу, не
дожидаясь остальных вопросов. Вы должны реализовать следующие две функции:

\begin{itemize}
\item \texttt{void initialize(int n)}~--- в начале будет вызвана функция \texttt{initialize}. Параметр $n$ задает число городов.
\item \texttt{int hasEdge(int u, int v)}~--- потом будет вызвана функция \texttt{hasEdge} $r = \frac{n(n - 1)}{2}$ раз. Каждый вызов этой функции~--- это очередной вопрос Мей-Ю. Вы должны выбрать ответ: имеется ли прямой авиарейс между городами $u$ и $v$. В частности, возвращаемое значение должно быть равно $1$, если прямой авиарейс есть, и $0$~--- в противном случае.
\end{itemize}